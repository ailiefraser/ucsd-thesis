%
%
% UCSD Doctoral Dissertation Template
% -----------------------------------
% https://github.com/ucsd-thesis/ucsd-thesis
%
%
% ----------------------------------------------------------------------
% WARNING: 
%
%   This template has not endorced by OGS or any other official entity.
%   The official formatting guide can be obtained from OGS.
%   It can be found on the web here:
%   http://grad.ucsd.edu/_files/academic-affairs/Dissertations_Theses_Formatting_Manual.pdf
%
%   No guaranty is made that this LaTeX class conforms to the official UCSD guidelines.
%   Make sure that you check the final document against the Formatting Manual.
%  
%   That being said, this class has been routinely used for successful 
%   publication of doctoral theses.  
%
%   The ucsd.cls class files are only valid for doctoral dissertations.
%
%
% ----------------------------------------------------------------------
% GETTING STARTED:
%
%   Lots of information can be found on the project wiki:
%   http://code.google.com/p/ucsd-thesis/wiki/GettingStarted
%
%
%   To make a pdf from this template use the command:
%     pdflatex template
%
%
%   To get started please read the comments in this template file 
%   and make changes as appropriate.
%
%   If you successfully submit a thesis with this package please let us
%   know.
%
%
% ----------------------------------------------------------------------
% KNOWN ISSUES:
%
%   Currently only the 12pt size conforms to the UCSD requirements.
%   The 10pt and 11pt options make the footnote fonts too small.
%
%
% ----------------------------------------------------------------------
% HELP/CONTACT:
%
%   If you need help try the ucsd-thesis google group:
%   http://groups.google.com/group/ucsd-thesis
%
%
% ----------------------------------------------------------------------
% BUGS:
%
%   Please report all bugs at:
%   https://github.com/ucsd-thesis/ucsd-thesis/issues
%
%
% ----------------------------------------------------------------------
% More control of the formatting of your thesis can be achieved through
% modifications of the included LaTeX class files:
%
%   * ucsd.cls    -- Class file
%   * uct10.clo   -- Configuration files for font sizes 10pt, 11pt, 12pt
%     uct11.clo                            
%     uct12.clo
%
% ----------------------------------------------------------------------



% Setup the documentclass 
% default options: 12pt, oneside, final
%
% fonts: 10pt, 11pt, 12pt -- are valid for UCSD dissertations.
% sides: oneside, twoside -- note that two-sided theses are not accepted 
%                            by OGS.
% mode: draft, final      -- draft mode switches to single spacing, 
%                            removes hyperlinks, and places a black box
%                            at every overfull hbox (check these before
%                            submission).
% chapterheads            -- Include this if you want your chapters to read:
%                              Chapter 1
%                              Title of Chapter
%
%                            instead of
%                              1 Title of Chapter
\documentclass[12pt,chapterheads]{ucsd}



% Include all packages you need here.  
% Some standard options are suggested below.
%
% See the project wiki for information on how to use 
% these packages. Other useful packages are also listed there.
%
%   http://code.google.com/p/ucsd-thesis/wiki/GettingStarted



%% AMS PACKAGES - Chances are you will want some or all 
%    of these if writing a dissertation that includes equations.
%  \usepackage{amsmath, amscd, amssymb, amsthm}

%% GRAPHICX - This is the standard package for 
%    including graphics for latex/pdflatex.
\usepackage{scrextend}
\usepackage{pslatex}
\usepackage{graphicx}

%% CAPTION
% This overrides some of the ugliness in ucsd.cls and
% allows the text to be double-spaced while letting figures,
% tables, and footnotes to be single-spaced--all OGS requirements.
% NOTE: Must appear after graphics and ams math
\makeatletter
\gdef\@ptsize{2}% 12pt documents
\let\@currsize\normalsize
\makeatother
\usepackage{setspace}
\doublespace
\usepackage[font=small, width=0.9\textwidth]{caption}

%% SUBFIG - Use this to place multiple images in a
%    single figure.  Subfig will handle placement and
%    proper captioning (e.g. Figure 1.2(a))
% \usepackage{subfig}

%% TIMES FONT - replacements for Computer Modern
%%   This package will replace the default font with a
%%   Times-Roman font with math support.
% \usepackage[T1]{fontenc}
% \usepackage{mathptmx}
\usepackage{fontspec}
\setmainfont{ACaslonPro-Regular.otf}[
    BoldFont = ACaslonPro-Bold.otf,
    ItalicFont = ACaslonPro-Italic.otf,
    BoldItalicFont = ACaslonPro-BoldItalic.otf
]

%% INDEX
%   Uncomment the following two lines to create an index: 
% \usepackage{makeidx}
% \makeindex
%   You will need to uncomment the \printindex line near the
%   bibliography to display the index.  Use the command
% \index{keyword} 
%   within the text to create an entry in the index for keyword.
%   To compile a LaTeX document with an index the 'makeindex'
%   command will need to be run.  See the wiki for more details.

%% HYPERLINKS
%   To create a PDF with hyperlinks, you need to include the hyperref package.
%   THIS HAS TO BE THE LAST PACKAGE INCLUDED!
%   Note that the options plainpages=false and pdfpagelabels exist
%   to fix indexing associated with having both (ii) and (2) as pages.
%   Also, all links must be black according to OGS.
%   See: http://www.tex.ac.uk/cgi-bin/texfaq2html?label=hyperdupdest
%   Note: This may not work correctly with all DVI viewers (i.e. Yap breaks).
%   NOTE: hyperref will NOT work in draft mode, as noted above.
% \usepackage[colorlinks=true, pdfstartview=FitV, 
%             linkcolor=black, citecolor=black, 
%             urlcolor=black, plainpages=false,
%             pdfpagelabels]{hyperref}
% \hypersetup{ pdfauthor = {Your Name Here}, 
%              pdftitle = {The Title of The Dissertation}, 
%              pdfkeywords = {Keywords for Searching}, 
%              pdfcreator = {pdfLaTeX with hyperref package}, 
%              pdfproducer = {pdfLaTeX} }
% \urlstyle{same}
% \usepackage{bookmark}


%% CITATIONS
% Sets citation format
% and fixes up citations madness
\usepackage{microtype}  % avoids citations that hang into the margin


%% FOOTNOTE-MAGIC
% Enables footnotes in tables, re-referencing the same footnote multiple times.
\usepackage{footnote}
\makesavenoteenv{tabular}
\makesavenoteenv{table}


%% TABLE FORMATTING MADNESS
% Enable all sorts of fun table tricks
\usepackage{rotating}  % Enables the sideways environment (NCPW)
\usepackage{array}  % Enables "m" tabular environment http://ctan.org/pkg/array
\usepackage{booktabs}  % Enables \toprule  http://ctan.org/pkg/array



\begin{document}

%% FRONT MATTER
%
%  All of the front matter.
%  This includes the title, degree, dedication, vita, abstract, etc..
%  Modify the file template_frontmatter.tex to change these pages.
%
%
% UCSD Doctoral Dissertation Template
% -----------------------------------
% http://ucsd-thesis.googlecode.com
%
%


%% REQUIRED FIELDS -- Replace with the values appropriate to you

% No symbols, formulas, superscripts, or Greek letters are allowed
% in your title.
\title{Contextually Recommending Expert Help and Demonstrations to Improve Creativity}

\author{Cristin Ailidh Fraser}
\degreeyear{\the\year}

% Master's Degree theses will NOT be formatted properly with this file.
\degreetitle{Doctor of Philosophy}

\field{Computer Science}
% \specialization{Anthropogeny}  % If you have a specialization, add it here

\chair{Scott R. Klemmer}
% Uncomment the next line iff you have a Co-Chair
% \cochair{Professor Cochair Semimaster}
%
% Or, uncomment the next line iff you have two equal Co-Chairs.
%\cochairs{Professor Chair Masterish}{Professor Chair Masterish}

%  The rest of the committee members  must be alphabetized by last name.
\othermembers{
Mira Dontcheva\\
William G. Griswold\\
Philip J. Guo\\
Björn Hartmann\\
James D. Hollan
}
\numberofmembers{6} % |chair| + |cochair| + |othermembers|


%% START THE FRONTMATTER
%
\begin{frontmatter}

%% TITLE PAGES
%
%  This command generates the title, copyright, and signature pages.
%
\makefrontmatter

%% DEDICATION
%
%  You have three choices here:
%    1. Use the ``dedication'' environment.
%       Put in the text you want, and everything will be formated for
%       you. You'll get a perfectly respectable dedication page.
%
%
%    2. Use the ``mydedication'' environment.  If you don't like the
%       formatting of option 1, use this environment and format things
%       however you wish.
%
%    3. If you don't want a dedication, it's not required.
%
%
% \begin{dedication}
%   Dedicated to... coming soon.
% \end{dedication}


% \begin{mydedication} % You are responsible for formatting here.
%   \vspace{1in}
%   \begin{flushleft}
% 	To me.
%   \end{flushleft}
%
%   \vspace{2in}
%   \begin{center}
% 	And you.
%   \end{center}
%
%   \vspace{2in}
%   \begin{flushright}
% 	Which equals us.
%   \end{flushright}
% \end{mydedication}



%% EPIGRAPH
%
%  The same choices that applied to the dedication apply here.
%
% \begin{epigraph} % The style file will position the text for you.
%   \emph{A careful quotation\\
%   conveys brilliance.}\\
%   ---Smarty Pants
% \end{epigraph}

% \begin{myepigraph} % You position the text yourself.
%   \vfil
%   \begin{center}
%     {\bf Think! It ain't illegal yet.}
%
% 	\emph{---George Clinton}
%   \end{center}
% \end{myepigraph}


%% SETUP THE TABLE OF CONTENTS
%
\tableofcontents
\listoffigures  % Comment if you don't have any figures
\listoftables   % Comment if you don't have any tables



%% ACKNOWLEDGEMENTS
%
%  While technically optional, you probably have someone to thank.
%  Also, a paragraph acknowledging all coauthors and publishers (if
%  you have any) is required in the acknowledgements page and as the
%  last paragraph of text at the end of each respective chapter. See
%  the OGS Formatting Manual for more information.
%
\begin{acknowledgements}
 I am grateful to Scott and Mira for their endless patience, encouragement, support, and wisdom. Thank you for believing in my research even when I didn't. When I first visited UCSD, Scott took me for a walking meeting through the gorgeous eucalyptus forest. Not only was this a genius recruiting tactic, it gave me a first glimpse into his unique approach to research, communication, and life. I've learned so much more from Scott than I realize even now, and am ever grateful for the tight-knit community he fostered in our lab and amongst his students. Mira and I first met in a North Toronto coffee shop, and she has continued to awe and inspire me ever since with her incredible mind, generous heart, and endless passion for her work. It has been such a joy to work with Mira and I could not be more excited to continue doing so at Adobe. I'm rarely sure of anything in my life, but I am sure that this next step is exactly where I want to go.

I am lucky to have such a talented and supportive committee. Thanks to Jim for always giving me such positive and encouraging feedback and guidance, and always being open to a chat. Thanks to Philip for reminding me why my research is exciting when I lose steam, and for so much valuable advice over the years. Thank you to Bill for many stimulating conversations about research, photography, and life. And finally, thanks to Björn for guiding my work with insightful questions and thoughtful feedback.

The Design Lab and CSE Department have both fostered wonderful communities that I am grateful to be a part of. I am thankful to Don Norman for his leadership of the Design Lab and for welcoming me on that first day back in 2014 when I shyly walked into the lab. Thank you to the many members of the Design Lab who have given me so much valuable feedback. Thanks to Julia and James for being such wonderful mentees and collaborators, and for giving me the joy of seeing their passion for research grow. Thanks to Ian, Vanessa, Sara, Olga, and the entire Design Lab operations team past and present, for their indispensable support. And Teenah, who has supported me with her endless generosity since the very start: it's been a delight to learn and grow alongside you. Thank you to Julie for always keeping me on track, and for the many quick meetings that turned into long conversations. 
I am grateful for funding from UCSD's Powell Fellowship, NSERC, Adobe Research, and the CSE Department.


I could not have gotten through this journey without an incredible support system of labmates, collaborators, and friends. Thank you to Cat for patiently teaching me the ropes and welcoming me with open arms into your San Diego world. Thank you to Vineet, Tricia, Ariel, and the rest of the crewtons for sushiboocha, coffee breaks, tea time, naps in the lab, late night deadline companionship, and everything else. Thank you Vineet for paving the way ahead, keeping my spirits up in our windowless office, and testing all your good (and bad) jokes on me. I'll always remember those three magical months when we got to work in the office with windows. Tricia, my two-time-award-winning co-author, thank you for keeping me sane through our research projects (especially the studies) and for always being awkward with me. Thank you Amy for all our adventures, and for the world's best napping chair. Thank you Tiffany for your life-changing support throughout this journey, and Vivian for going above and beyond in your care.

Thank you Ariana for being an exceptional roommate, friend, and colleague, and for your endless support and energy. In the time you have been here, you have transformed not only my life, but also GradWIC and the entire CSE Department. Thanks to all those who have contributed to GradWIC for helping us build an important and ever-growing community. Thank you to the many aerial friends and mentors I've made at UCSD and AR, especially Chava and her Angells, for getting me in better shape (physically and mentally) than I've ever been. Thank you Diana for following me to San Diego (sorry I'm leaving now) and including me in all your family adventures. Thank you Chris for growing with me this entire way, supporting me through the worst times, and celebrating with me in the best times. You make me a better person, and I can't wait to continue this adventure together.

I am grateful to Adobe and Autodesk for supporting me through four summer internships. To have spent one summer reading all of Tovi Grossman and George Fitzmaurice's papers, and then the following summer writing a paper with both of them was an absolute dream. I am grateful to have worked with and learned from so many exceptional researchers: Joel Brandt, Joy Kim, Valentina Shin, Holger Winnemöller, Sheryl Ehrlich, Andy Wilson, Alison Thornsberry, Fraser Anderson, and others. Thanks to the many fellow interns I've shared these experiences with, especially the Miracles (Yea-Seul, Jasper, Amanda, Daniel) and my Canadian buddies, David and Rahul. Thanks to Minsuk, who I've somehow never actually done an internship with but who was never too far away. Possibly one of the most successful meetups enabled by Confer! Thanks to Jordana, Rachel, and Willy for the luckiest summer roommate situation.

Diane Horton's passion for teaching and constant encouragement is what first got me interested in Computer Science, and for that I will be forever grateful. Thanks also to Karen Reid and Paul Gries for their incredible teaching and mentorship. Thank you to Kyros Kutulakos and Matt O'Toole for showing me how to do good research. Thank you to my former office-mate and soon-to-be colleague Peter O'Donovan for introducing me to HCI and to Scott's research.

And finally, thanks to my family. To my mother Nancy for all the crucial stats help, my father Don for telling people I was doing a PhD before I'd even decided to, my sister Donelle for her never-ending kindness and generosity, and of course Sammy and Tibby.
\\

\textsc{Chapter \ref{chapter:replay}}, in part, includes portions of material as it appears in \textit{RePlay: Contextually Presenting Learning Videos Across Software Applications} by C. Ailie Fraser, Tricia J. Ngoon, Mira Dontcheva, and Scott Klemmer in the Proceedings of the 2019 CHI Conference on Human Factors in Computing Systems (CHI '19). The dissertation author was the primary investigator and author of this paper.

\textsc{Chapter \ref{chapter:remap}}, in part, is currently being prepared for submission for publication of the material. C. Ailie Fraser, Julia M. Markel, N. James Basa, Mira Dontcheva, and Scott Klemmer. The dissertation author was the primary investigator and author of this material.

\textsc{Chapter \ref{chapter:liveclips}}, in part, includes portions of material as it appears in \textit{Sharing the Studio: How Creative Livestreaming can Inspire, Educate, and Engage} by C. Ailie Fraser, Joy O. Kim, Alison Thornsberry, Scott Klemmer, and Mira Dontcheva in the Proceedings of the 2019 on Creativity and Cognition (C\&C '19). The dissertation author was the primary investigator and author of this paper.

\textsc{Chapter \ref{chapter:liveclips}}, in part, includes portions of material coauthored with Andy Edmonds and Mira Dontcheva. The dissertation author was the primary investigator and author of this material.

\textsc{Chapter \ref{chapter:discoveryspace}}, in part, includes portions of material as it appears in \textit{DiscoverySpace: Suggesting Actions in Complex Software} by C. Ailie Fraser, Mira Dontcheva, Holger Winnemöller, Sheryl Ehrlich, and Scott Klemmer in the Proceedings of the 2016 ACM Conference on Designing Interactive Systems (DIS '16). The dissertation author was the primary investigator and author of this paper.

\textsc{Chapter \ref{chapter:critiquekit}}, in part, includes portions of material as it appears in \textit{Interactive Guidance Techniques for Improving Creative Feedback} by Tricia J. Ngoon, C. Ailie Fraser, Ariel S. Weingarten, Mira Dontcheva, and Scott Klemmer in the Proceedings of the 2018 CHI Conference on Human Factors in Computing Systems (CHI '18). The dissertation author was one of the primary investigators and authors of this paper.

\textsc{Appendix \ref{chapter:appendix}}, in part, includes portions of material as it appears in \textit{Sharing the Studio: How Creative Livestreaming can Inspire, Educate, and Engage} by C. Ailie Fraser, Joy O. Kim, Alison Thornsberry, Scott Klemmer, and Mira Dontcheva in the Proceedings of the 2019 on Creativity and Cognition (C\&C '19). The dissertation author was the primary investigator and author of this paper.
\end{acknowledgements}


%% VITA
%
%  A brief vita is required in a doctoral thesis. See the OGS
%  Formatting Manual for more information.
%
\begin{vitapage}
\begin{vita}
  \item[2013] Honours B.Sc. with High Distinction, Specialist in Math \& Computer Science, Major in Music, University of Toronto, Canada
  \item[2016] M.S. in Computer Science, University of California San Diego
  \item[2020] Ph.~D. in Computer Science, University of California San Diego
\end{vita}
\begin{publications}
\item C. Ailie Fraser, Joy Kim, Valentina Shin, Joel Brandt, and Mira Dontcheva. 2020. Temporal Segmentation of Creative Live Streams. To appear in \textit{Proceedings of the 2020 CHI Conference on Human Factors in Computing Systems (CHI '20)}.
\item C. Ailie Fraser, Mira Dontcheva, Joy O. Kim, and Scott Klemmer. 2019. How live streaming does (and doesn't) change creative practices. \textit{interactions} 27, 1 (December 2019), 46–51.
\item C. Ailie Fraser, Joy O. Kim, Alison Thornsberry, Scott Klemmer, and Mira Dontcheva. 2019. Sharing the Studio: How Creative Livestreaming can Inspire, Educate, and Engage. In \textit{Proceedings of the 2019 on Creativity and Cognition (C\&C '19)}. Association for Computing Machinery, New York, NY, USA, 144–155.
\item C. Ailie Fraser, Tricia J. Ngoon, Mira Dontcheva, and Scott Klemmer. 2019. RePlay: Contextually Presenting Learning Videos Across Software Applications. In \textit{Proceedings of the 2019 CHI Conference on Human Factors in Computing Systems (CHI '19)}. Association for Computing Machinery, New York, NY, USA, Paper 297, 1–13.
\item Tricia J. Ngoon, C. Ailie Fraser, Ariel S. Weingarten, Mira Dontcheva, and Scott Klemmer. 2018. Interactive Guidance Techniques for Improving Creative Feedback. In \textit{Proceedings of the 2018 CHI Conference on Human Factors in Computing Systems (CHI '18)}. Association for Computing Machinery, New York, NY, USA, Paper 55, 1–11.
\item C. Ailie Fraser, Tovi Grossman, and George Fitzmaurice. 2017. WeBuild: Automatically Distributing Assembly Tasks Among Collocated Workers to Improve Coordination. In \textit{Proceedings of the 2017 CHI Conference on Human Factors in Computing Systems (CHI '17)}. Association for Computing Machinery, New York, NY, USA, 1817–1830.
\item C. Ailie Fraser, Mira Dontcheva, Holger Winnemöller, Sheryl Ehrlich, and Scott Klemmer. 2016. DiscoverySpace: Suggesting Actions in Complex Software. In \textit{Proceedings of the 2016 ACM Conference on Designing Interactive Systems (DIS '16)}. Association for Computing Machinery, New York, NY, USA, 1221–1232.
\item Catherine M. Hicks, Vineet Pandey, C. Ailie Fraser, and Scott Klemmer. 2016. Framing Feedback: Choosing Review Environment Features that Support High Quality Peer Assessment. In \textit{Proceedings of the 2016 CHI Conference on Human Factors in Computing Systems (CHI '16)}. Association for Computing Machinery, New York, NY, USA, 458–469.
\end{publications}
\end{vitapage}


%% ABSTRACT
%
%  Doctoral dissertation abstracts should not exceed 350 words.
%   The abstract may continue to a second page if necessary.
%
\begin{abstract}
  Translating goals into concrete actions is a major challenge for people doing creative activities. Expert examples, tutorials, and videos abound online, helping people find inspiration and learn from the best. However, online resources are detached from the user’s work. The user must determine which resource is most relevant to their particular situation, and adapt it to their context. The sensemaking challenge of using help resources combined with the open-endedness of creative work can be tedious at best, and paralyzing at worst, preventing people from reaching their full creative potential.

My dissertation introduces methods for \textbf{curating existing expert help and demonstrations} and \textbf{presenting them to users in context}, with the goal of better supporting the iterative creative process and thus enabling people to produce better work. Leveraging existing expert-made resources enables people to learn from others more easily, and placing them in context promotes serendipity and unexpected connections.

When performing creative activities, people can be driven by processes and/or outcomes: sometimes they are interested in understanding the process behind creative tasks, while other times they wish to reach a particular outcome the fastest and easiest way possible. This dissertation presents four interactive systems that explore how three different types of expert resource -- \textbf{visual media}, \textbf{executable code}, and \textbf{written text} -- can help people accomplish one or both of the above goals at a level beyond what they could have achieved otherwise.

Specifically, \textit{RePlay} and \textit{ReMap} support both process and outcome by enabling in-task search and navigation of help videos. \textit{LiveClips} supports discovery and exploration of expert processes by embedding inspirational clips from live streamed videos into the user’s workflow. \textit{DiscoverySpace} helps novices discover and achieve expert outcomes by recommending action macros in-situ. Finally, \textit{CritiqueKit} combines adaptive suggestions with interactive guidance to help novices improve their outcomes in the moment.

Together, these systems and their evaluations demonstrate how curating and presenting expert help and demonstrations in-situ can help novices build confidence, accomplish tasks, and produce higher-quality creative work. My dissertation enables more people to reach their creative potential by lowering the barriers to getting started and completing projects.

\end{abstract}


\end{frontmatter}






%% DISSERTATION

% A common strategy here is to include files for each of the chapters. I.e.,
% Place the chapters is separate files: 
%   chapter1.tex, chapter2.tex
% Then use the commands:
%   \include{chapter1}
%   \include{chapter2}
%
% Of course, if you prefer, you can just start with
%   \chapter{My First Chapter Name}
% and start typing away.  
\chapter{Just a Test}
This is only a test.
\section{A section}
Lorem ipsum dolor sit amet, consectetuer adipiscing elit. Nulla odio
sem, bibendum ut, aliquam ac, facilisis id, tellus. Nam posuere pede
sit amet ipsum. Etiam dolor. In sodales eros quis pede.  Quisque sed
nulla et ligula vulputate lacinia. In venenatis, ligula id semper
feugiat, ligula odio adipiscing libero, eget mollis nunc erat id orci.
Nullam ante dolor, rutrum eget, vestibulum euismod, pulvinar at, nibh.
In sapien. Quisque ut arcu. Suspendisse potenti. Cras consequat cursus
nulla.

\subsection{A Figure Example}
\label{ssec:figure_example}

This subsection shows a sample figure.

\begin{figure}[h] 
  \centering
  \includegraphics[width=0.5\textwidth]{sandiego}
  \caption[A picture of San Diego. Short figure caption must be \protect{$< 4$} lines in the list of figures]
{A picture of San Diego.  Short figure caption must be \protect{$< 4$} lines in the list of figures and match the start of the main figure caption verbatim. Note that figures must be on their own line (no neighboring text) and captions must be single-spaced and appear \protect\textit{below} the figure.  Captions can be as long as you want, but if they are longer than 4 lines in the list of figures, you must provide a short figure caption.\index{SanDiego}}
  \label{fig:sandiego}
\end{figure}

\subsection{A Table Example}

While in Section \ref{ssec:figure_example} Figure \ref{fig:sandiego} we had a majestic figure, here we provide a crazy table example.


%%%% TABLE 1 %%%%
\vspace{0.25in}
\begin{table}[!ht]
\caption[A table of when I get hungry.  Short table caption must be \protect{$< 4$} lines in the list of tables]{A table of when I get hungry. Short table caption must be \protect{$< 4$} lines in the list of tables and match the start of the main table caption verbatim.  Note that tables must be on their own line (no neighboring text) and captions must be single-spaced and appear \protect\textit{above} the table.  Captions can be as long as you want, but if they are longer than 4 lines in the list of figures, you must provide a short figure caption.}

\vspace{-0.25in}
\begin{center}
\begin{tabular}{|p{1in}|p{2in}|p{3in}|}

\hline
Time of day & Hunger Level & Preferred Food \\

\hline
8am & high & IHOP (French Toast) \\

\hline
noon & medium & Croutons (Tomato Basil Soup \& Granny Smith Chicken Salad) \\

\hline
5pm & high & Bombay Coast (Saag Paneer) or Hi Thai (Pad See Ew) \\

\hline
8pm & medium & Yogurt World (froyo!) \\

\hline
\end{tabular}
\end{center}
\label{tab:analysis3}
\end{table}



%% APPENDIX
\appendix
\chapter{Final notes}
What to do about things \cite{Martin_1983}.  What did he say \cite{Rilling_Insel_1999}.
  Remove me in case of abdominal pain.



%% END MATTER
% \printindex %% Uncomment to display the index
% \nocite{}  %% Put any references that you want to include in the bib 
%               but haven't cited in the braces.
\bibliographystyle{alpha}  %% This is just my personal favorite style. 
%                              There are many others.
%\setlength{\bibleftmargin}{0.25in}  % indent each item
%\setlength{\bibindent}{-\bibleftmargin}  % unindent the first line
%\def\baselinestretch{1.0}  % force single spacing
%\setlength{\bibitemsep}{0.16in}  % add extra space between items
\bibliography{template}  %% This looks for the bibliography in template.bib 
%                          which should be formatted as a bibtex file.
%                          and needs to be separately compiled into a bbl file.
\singlespace  %to force bibilography environment to use single spacing for each entry 
              %double spacing between entries remains
\end{document}

