\section{Discussion}
Despite the relatively small sample sizes when participants are broken up into sub-conditions, our results suggest that action suggestions can be beneficial for novice users of complex software. Further research should explore these possible effects in more detail.

It may not be that surprising that the confidence levels of participants in the Control condition dropped after using Photoshop. Novice participants were quickly frustrated with the time it took to find functionality and get accustomed to the interface, and more experienced participants struggled to locate forgotten tools and commands. We believe that Discovery\-Space, while it may not currently support all possible goals, provided a friendlier starting point that showed or reminded participants what types of edits are possible. Again, it is not that surprising that this effect was more pronounced for Beginners, as they are the most likely to misjudge their abilities, having never or rarely used Photoshop.

We assume that participants who answered ``no'' to the question, ``Was there anything you wished you could do to this photo but couldn't figure out how?'' were able to accomplish the tasks they set out to do. Participants who answered ``yes'' to the question were prompted to describe what they could not figure out. Most responses named a task the participant had been trying to accomplish (\textit{e.g.}, removing redness from a person's face), but either gave up on or ran out of time. It is likely that Discovery\-Space provided quicker ways to accomplish many of these tasks, and thus participants in the Discovery\-Space condition were more likely to complete them in the time allotted than those in the Control condition. 

As the study involved using the full Photoshop interface, expertise had a great impact on user performance. Observations during the sessions confirmed that Expert users were comfortable with Photoshop's interface, whereas Beginners tended to find it overwhelming because of the large number of menus and buttons. Ratings of creativity and confidence about performance varied widely, likely because the task was open-ended, and these measures depended on each individual's goal. 
The correlation between confidence in performance and creativity indicates that participants in both conditions did tend to include creativity as a criterion for doing a good job, supporting the motivation for an interface that encourages creative exploration. Participants who used the web achieved their goals less often, likely because these are users who had more trouble with their tasks and turned to the web for help. However, using the web did not in fact seem to help them. The following section provides some suggestions as to why that may be.

\subsection{Common Difficulties}
The same researcher conducted all 28 sessions, and ob-served participants' recurring errors and difficulties. The most prominent observations, outlined here, support our motivation for action suggestions, and provide directions for future work.

\subsubsection{1. Failing to find the best way to accomplish a task}
Several participants used Google to search for help and often followed the top result without realizing that there was an easier, more efficient, or more effective way to accomplish the task. For example, the automatically generated Google summary listed at the top for ``how to crop photoshop'' suggests using the Rectangular Marquee tool, rather than the more relevant Crop tool. In addition, many tutorials included screenshots from older versions of Photoshop, making it difficult for participants to find the tools mentioned. Version-specific help is a challenge for all software. Systems for augmenting search queries with the user's context could improve this (\textit{e.g.} [4]). For example, the top search result shown when ``cc 2015'' is appended to the above Google search is recent and demonstrates the Crop tool. Such functionality could be helpful for Discovery\-Space's search feature when a larger corpus of actions or tutorials is added.

\subsubsection{2. Not noticing when commands cause state changes or side effects}
A common issue participants ran into was creating new layers without realizing, and then having the wrong layer selected when trying to apply another effect. For example, performing an adjustment from the Adjustments panel creates a new Adjustment layer, but does not explicitly notify the user of this. Most actions in Discovery\-Space also create at least one layer. In this study, many participants in both conditions struggled with having the wrong layer selected when trying to apply other edits, which meant those edits did not work as expected. This is an instance of a more general problem in complex software: commands often have side effects that the user is not aware of, which may cause confusion later on. This suggests that programs should 1) make the user aware of what they are doing, and 2) encourage better development of a correct mental model of the program, so that such side effects are not so surprising when they occur. Support for better interactive history may be an important improvement for Discovery\-Space as it would allow participants to work through and explore the steps that have been completed.

\subsubsection{3. Difficulty finding out what tools do}
For novices, complex software has poor information scent [28]: we observed users sequentially browsing menus, toolbars and panels to find something specific or see what is possible. Even with exhaustive search, Beginners seemed to have difficulty understanding what the buttons and menu items did, given only their names and brief tooltip descriptions. When participants did find the desired tool, they often had trouble figuring out how to use it. Better in-app descriptions and assistance are needed. For example, ToolClips augments tool-tips to include media content and detailed information [11]. This is an advanced form of ``speaking navigation'': verbose navigational links that improve information scent. While the current goal of Discovery\-Space was to move away from individual tools toward better task-level support, one way to improve in-formation scent for tools could be to display example tasks that a tool is often used for when mousing over it.

\subsection{Study Limitations}
Having an open-ended task facilitated more realistic usage than a pre-defined task, and having participants work on their own personal photos likely increased their motivation to do a good job. However, participants' widely varying goals resulted in widely varying results regarding their editing experience and likelihood of achieving their goals. Future studies might consider more directed tasks to reduce the variability in participants' goals. 

Using self-reported expertise ratings has the potential for inaccurate assignments based on differences in how participants perceive their own skill. Our observations found their self-report sufficiently accurate. However, future studies might consider a more objective expertise measure, such as a pre-test. The pool of ``novices'' is larger than one might think: many Photoshop novices were experienced with photography principles and/or simpler photo editing soft-ware. They had domain knowledge, just not expertise on the specific task at hand. Especially with feature-rich soft-ware like Photoshop, nearly everyone is a novice in some regard. 

Participants' widely varying expertise levels yielded small sample sizes for measuring interaction effects. However, for this preliminary study, this variety unearthed valuable differences in participant behaviour. 

Finally, many participants ran out of time and stated this as the main reason they could not achieve their goal; future studies should allow participants to work for longer. A longitudinal study would allow for more realistic usage data, as participants could use the application on their own time and more goals could be collected per person. 
