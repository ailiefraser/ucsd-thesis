\section{Future Work}
Participants' experiences with Discovery\-Space suggest several ways to improve the robustness and usability of action suggestions in complex software.

\subsection{Improving Discovery\-Space for Photo Editing}
An important limitation of the current Discovery\-Space prototype is that most actions in its corpus produce \textit{global} edits, not \textit{local} ones. A few actions intended for faces (\textit{e.g.} skin smoothing) allow the user to brush on the effect, but the majority apply to the entire photo. This is representative of the selection of free Photoshop actions available online; most are global edits because these are much easier to create than ones that allow for user input such as selection. However, users often desire local editing capabilities, and so providing actions that allow for this would be valuable. One way of providing support for user input would be to leverage the interactive steps provided by TappCloud \cite{Laput2012}.

Based on participants' responses to the question of what tasks they would use Discovery\-Space for, it seems to be effective for quick exploration. In order to help users make more professional-looking edits, Discovery\-Space could take better advantage of the properties of the user's input photo and apply effects in a content-adaptive way \cite{Berthouzoz2011} or suggest automatic fixes based on properties of the photo such as its histogram (\textit{e.g.} as in \cite{Bychkovsky2011}). 

Other immediate improvements to Discovery\-Space for photo editing will include automatic image analysis to replace manual user descriptions of photos, and replacing the default preview images of effects with previews of the user's actual photo.

\subsection{General Improvements to Action Suggestions}
Many programs have the capability to record and save actions (\textit{e.g.}, Adobe Photoshop, Illustrator, and Acrobat), also referred to as presets (\textit{e.g.}, Adobe Lightroom), action macros (\textit{e.g.}, Autodesk's AutoCAD), and macros (\textit{e.g.}, Microsoft Excel and Word). Even more programs allow for task automation via scripting (\textit{e.g.}, Adobe InDesign, Mac OS). Discovery\-Space and the following proposed improvements can apply not only to Photoshop but to any complex software that provides some way of combining operations into actions.

\subsubsection{End-user control}
The most frequent piece of feedback received throughout the study and prototyping process was the desire for more control over the effects. One of the main advantages complex programs have over simple ones (\textit{e.g.}, Instagram) is the ability to have fine-grained control over operations, and so providing this control to users would be a valuable component of an interface like Discovery\-Space. This could be accomplished by examining current usage data of the software in question to determine what parameters users most often edit for common operation sequences, and making only those parameters available when an action is executed. Alternatively, Discovery\-Space could make the action history more interactive by allowing users to select, edit, and delete any step that was performed. To provide an intuitive way of setting parameters, visual previews like the variations in Side Views \cite{Terry2002} and TappCloud \cite{Laput2012} could be incorporated. 

\subsubsection{Recommendation algorithm}
Given the diverse goals that users have in complex applications, a technique like collaborative filtering could improve recommendations by personalizing the suggestions to the user and their goals. Li et al. recommend an ``item-based'' collaborative filtering algorithm based on their work with CommunityCommands \cite{Li2011}. This algorithm bases recommendations on the similarity between commands, rather than between users. They also found that users preferred recommendations based on their behaviour within the current session, rather than long-term. The goals users enter into Discovery\-Space should also be included as input to the suggestion algorithm, so that suggestions are not only relevant to the user's photo, but also to the goal they have in mind. This natural language description could also be used to annotate the actions taken by users. With this extra information, we could build a better model of the relationships between goals, the language used to express them, and the tools or actions used. Such descriptions would otherwise be difficult to collate, as users would be required to do extra work to describe the actions they perform.

\subsubsection{Building a scalable corpus of actions}
User-generated actions are posted across multiple sites and carry differing amounts of metadata and detail, and differing distribution licenses. Curating our corpus of actions required careful attention to these differences, and we had to manually define keywords and names for most actions. Scaling up such a corpus or building it automatically would require accounting for these issues. This could be done by creating a centralized repository inside the software itself or as part of Discovery\-Space, wherein users can create and share actions, and are required to provide keywords and agree to a distribution license. An existing example of this is AdaptableGIMP, an interface for the photo editor GIMP that provides user-created actions referred to as ``task sets'' \cite{Lafreniere2011}. Users can create and share task sets from within the interface, and search the corpus for ones to apply.

Action suggestions need not be user-created. They could instead be generated by automatically recording and segmenting user behaviour along with the type of document being worked on, or by mining and analyzing the large collection of online tutorials that are available for most applications. Work has been done on generating interactive tutorials from static online ones \cite{Fourney2014Mining, Laput2012}, and incorporating such work into a search and suggestion interface like Discovery\-Space is a promising direction for improving both the initial usability and long-term learnability of complex software applications.
