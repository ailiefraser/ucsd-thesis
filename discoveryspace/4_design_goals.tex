\section{Design Goals}
Presenting suggestions relevant to the user's goal can help them accomplish it, as the experts did. Presenting suggestions the user might not have otherwise thought of can help them to discover what is possible and achieve creative results. Combining these insights with prior recommendation research (\textit{e.g.} \cite{Hearst2006, Hearst2009Book, Koren2008, Li2011}), we developed five main \textbf{design goals} for suggestion interfaces:
\begin{enumerate}
    \item \textit{Help users get started}: Make suggestions available as soon as the user begins a task. A frequent observation throughout our formative study and prototyping process was that users often did not know where to start in Photoshop.
    \item \textit{Use human language}: Allow users to search using goal terminology, and describe suggestions in a language novices can understand by including a descriptive non-technical title for each. This reflects the growing popularity of natural language or ``semantic'' search \cite{Hearst2009Book}.
    \item \textit{Show previews} of what a suggestion does, and allow users to easily compare before and after, for example as in Side Views \cite{Terry2002}.
    \item \textit{Offer faceted browsing} in addition to search to help users explore. This is a well-documented concept in the information retrieval literature (\textit{e.g.}, \cite{Hearst2006, Koren2008, Yee2003}), and we believe it to be applicable to software applications as well.
    \item \textit{Suggestions should be relevant} to the user's current task, but should also alert the user to new or unknown possibilities. CommunityCommands users were found to prefer ``contextual'' suggestions related to their short-term history \cite{Li2011}, and users of complex software often desire the ability to discover new features \cite{MM-gi2000}.
\end{enumerate}

We iterated on the Discovery\-Space design based on feedback from users with varying levels of Photoshop proficiency. Because Discovery\-Space harvested actions from multiple online sources, they had widely varying amounts of user interaction: some automatically applied effects, while others paused at key points with dialogs for user interaction. We removed such pauses and dialogs when present because they tended to lack sufficient accompanying explanation, and therefore confused some non-expert pilot testers. Users often desired basic adjustments such as brightness/contrast, exposure, and saturation. The actions in our collection generally comprised several steps rather than just one, and so did not include many of these basic edits. We manually created 11 actions for these basic adjustments, which would simply initialize the values to auto, and leave the sliders visible for the user to adjust as they wish.