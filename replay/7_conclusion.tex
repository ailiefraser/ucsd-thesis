\section{Conclusion}
This chapter introduced an application-independent approach for contextually presenting videos and a demonstration of this approach in the RePlay system. RePlay shows how system accessibility features and video captions can be used to detect context and search within videos in a flexible, domain-general way. Like curb cuts or closed captioning \cite{Rose2002}, RePlay demonstrates how accessibility features can provide universally-beneficial assistance. 
Expanding accessibility and increasing cross-application consistency through guidelines and enforcement would benefit everyone. It would also expand application tailoring, integration, and assistance with systems like RePlay. Two studies demonstrated that cross-application contextual video assistance helps users spend more time on their task and less time searching for help. We also observed how contextual assistance can sometimes be at odds with peoples' desire to explore and figure things out via trial-and-error, due to the perceived time and attention switching costs of searching for help.
%and that the context most easily accrued from software usage may not always be the most relevant. 
%Future work should investigate these challenges and examine how contextual help affects workflows in the real world through a longitudinal study. 
The next chapter builds on RePlay to explore how we might further reduce these costs to searching for help.
This work brings us one step closer to leveraging the wisdom of the web for personalized, just-in-time learning.

\section{Acknowledgments}
We are thankful to Michelle Lee for her assistance in conducting Study 2, and to all study participants for their time and feedback. This work was supported in part by NSERC and Adobe Research.

This chapter, in part, includes portions of material as it appears in \textit{RePlay: Contextually Presenting Learning Videos Across Software Applications} by C. Ailie Fraser, Tricia J. Ngoon, Mira Dontcheva, and Scott Klemmer in the Proceedings of the 2019 CHI Conference on Human Factors in Computing Systems (CHI '19). The dissertation author was the primary investigator and author of this paper.