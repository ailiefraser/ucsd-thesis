\section{Discussion \& Future Work}
Observations and feedback from the two studies suggest several opportunities for future work. 

\subsection{How Can Contextual Assistance aid Exploration?}
% assistance should not discourage exploration, but intervene when it is needed
% people explore because incremental cognitive load of exploration is small
% search is a one-time switch cost. if we can make the switching cost lower (by embedding in app and adding context) this should increase usage, and leaves people with more registers to focus on what they want to do.

Study 2 participants often preferred manually exploring when it would have been more efficient to search for help. The study's subjective, open-ended task may have encouraged such exploration. Only one participant used the best method to update the photo grid - using Adobe \textsc{xd}'s repeat grid feature to update all photos simultaneously - after learning about it in a video. Other participants used various less-efficient methods (\textit{e.g.}, un-grouping the repeat grid and manually adding and re-sizing photos). Because these methods achieved their desired goal, participants may not have thought to investigate whether a faster option existed.

Interface exploration and browser search each have shortcomings. Exploration is a common problem-solving strategy because it can be enjoyable, it builds on domain knowledge, and each individual action is low cost \cite{Lafreniere2014a, Rieman1996}. However, for novices especially, exploration is cumulatively slow, its learned knowledge is hard to integrate, and it induces a high cognitive load \cite{Tuovinen1999, Lafreniere2014a}. A limitation of learning unfamiliar domains via exploration is that people may settle for sub-optimal methods or strategies because they are unaware of a better alternative. Moving from an application to a web search can yield better results, but has a higher initial cost as it lurches people out of their task. We believe that contextual search has the potential to offer the benefits of both approaches without either of the drawbacks. 

Six participants in Study 2 did not search at all; most felt they could figure things out via exploration. One participant said they \textit{``felt like I could find it if I searched for it [in the interface], which I did. I was able to figure it out.''} Two other participants mentioned that they felt searching for help would be more time-consuming than trial-and-error exploration. One stated that he wanted to search for help, but \textit{``I knew I didn't have time. I wanted to complete the task so I just hacked it.''} If Study 2's result that contextual search reduces search time holds, these perceptions may change over time. RePlay's occasionally-slow loading times may have also affected this perception; prior work shows that a difference in latency of search results as small as 300ms can discourage people from searching \cite{Brutlag2009}.

How might contextual assistance encourage productive exploration while providing intervention when needed? While proactive support can be beneficial, a challenge is providing assistance without being too disruptive \cite{Matejka2011}. To minimize distractions, the RePlay interface mostly changes only in direct response to user input. However, novices may not even realize they need help, making proactive suggestions more valuable. For example, RePlay could automatically refresh video results when the user seems stuck, or suggest relevant queries when the user begins to search. Future work should examine how these alternatives might change people's behavior and workflows over time.
%will having assistance readily available change behavior over time? people have learned to guess their way to the solution and this is good in a lot of ways. but is there some interesting potential around a categorical strategy/behavior shift
% One Web participant said that he \textit{``almost always won't watch video because it's time-consuming...I usually prefer text-based help.''}. Similarly, another participant preferred text because \textit{``it's easier to parse rather than searching through a video.''} 

\subsection{What and How Much Context to Include?}
While logging recent tools can help suggest next steps \cite{Matejka2009}, we found that using recent tools explicitly in search queries is not useful. Participants in both studies did not use tool context in their queries, preferring action-oriented queries instead. Usage history is by definition retrospective (\textit{i.e.}, it describes what the user has already done). In contrast, search is often prospective (\textit{i.e.}, looking for something the user hasn't done yet). Tool context may only be helpful closer to where users interact with tools (\textit{e.g.,} as part of tooltips \cite{Grossman2010a}).
%can help divulge this information to users by increasing tools' information scent \cite{Pirolli2009}.

Interestingly, people used the same terms and concepts across different applications. Study 2 participants searched ``crop photos'' for both Canva and Adobe \textsc{xd} despite neither application having an explicit crop tool. This highlights both a challenge and an opportunity: people bring mental models that may not carry over into different applications. Study 1 suggested that displaying tool names may help people learn app-specific terminology. However, for both knowledge and speed reasons, users sometimes omit valuable terms. RePlay's interface benefits are reduced when users don't search using the same terms as videos. A natural language mapping \cite{Adar2014} between video captions (along with other natural language data like comments and tags) and the tools they mention may increase captions' value for search.

Finally, what other contextual information besides tool use might be helpful for identifying relevant videos? Liu \textit{et al.} \cite{Liu2020} showed that in addition to command logs, information about the use of layers and the time between interaction events can help segment usage logs from image editing tasks into subtasks. Similarly, information such as the name of the user's active layer might be helpful as additional context for a search query. Liu \textit{et al.} also found that the visual change and location of the user's edits were less useful for segmentation, but perhaps the visual content of the user's canvas or document could be used to identify videos with similar content. Future work should explore how we might collect and use such types of contextual information to search for help in an application-general way.

\subsection{What Design Challenges Remain?}
Many design decisions were motivated by our focus on cross-application workflows; \textit{e.g.,} showing results in a separate window in a consistent location. Showing all results together made it easy to browse multiple resources at once. However, some participants still preferred focusing on one video at a time. Future work should consider how different layouts influence browsing behaviors and which behaviors lead to more effective workflows.

Currently, users must explicitly whitelist an application for RePlay to capture its events. While this approach offers more privacy, it also adds burden for users.  Blacklisting, on the other hand, would allow RePlay to respond to all applications except for explicitly-omitted potentially-sensitive ones (\textit{e.g.,} Messages), offering broader benefits. For our initial studies, we chose the greater privacy of a whitelist. For real use, users could choose whitelist or blacklist, and/or RePlay could request approval for each new application, similar to websites that ask for a user's location.

\subsection{What Other Domains Might Benefit?}
RePlay's main insight is that given a source of user context, we can search, curate, and index into resources from a large corpus. RePlay demonstrates this approach using video; different activities (\textit{e.g.,} programming) may benefit from other types of content (\textit{e.g.,} text resources). RePlay could naturally be extended to any textual resource (or resource with textual metadata). Text results could be displayed as short summaries with clickable keywords to expand more detail~\cite{Ekstrand2011}. For detecting context, RePlay used Mac\-OS's accessibility \textsc{api}; other \textsc{os}s (\textit{e.g.} Windows \cite{Matejka2013}) also have similar \textsc{api}s. %Minor differences should not significantly affect RePlay's approach, as the user's query is still the primary search input.
Beyond software, RePlay's approach could extend to any domain for which online videos are abundant (\textit{e.g.,} physical building tasks). To detect activity context, one could augment physical tools with sensors \cite{Schoop2016, Lukowicz2004} or track body poses with wearable sensors or computer vision. A challenge for future work is to convert sensor or vision data into text searches, or to index videos using the sensor or vision data directly. 

% We used the application name, but the domain of the user's task may also be important for narrowing down results regarding multi-functional applications like Photoshop or Powerpoint.

% first study had a lot of retrospective context, was not helpful
% usage histor/state is retrospective and not always helpful for prospective search -- how address this without being clippy?
% eric horovitz
% how move forward with this and not end up with clippy? how to have good mixed initiative support?
% proactive variations: generate alternatives automatically and show them to you i already did it
% what about proactively suggesting search queries instead of videos/resources?

%All three participants in our field study almost exclusively typed their own search queries rather than using the ones RePlay provided. Using multiple recent tools explicitly in a search query takes the focus away from the specific question or tool the user wants help with. An alternative approach (suggested by \textit{P1}) could instead show recent tools as clickable elements for quick searches, while also providing a search box for user-generated queries.

% Future work: use prefab methods to get visual stuff too, could search with that

% \subsection{The First Result is Most Relevant for Context}
% Effective search interfaces should not only present relevant results, but also present the most relevant results at the top of the set [Hearst]. As we observed in Study 2, people tend to habitually choose the first result provided, expecting it to contain the most relevant information needed.
