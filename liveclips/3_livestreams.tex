\section{Understanding creative live-streams}
Creative live-streams are a unique and rapidly growing source of data that have yet to be deeply studied. While some work has examined live-streaming in the context of video gaming (e.g., \cite{Pan2016, Hamilton2014, Sjoblom2017}), to our knowledge there is no work exploring the growing trend of live-streaming creative projects. To understand the current landscape of creative live-streams as well as their applicability to inspiration in software, we explored and watched videos on two popular platforms, Twitch and Youtube. We analyzed videos by 47 different artists working on creative projects such as digital illustration, graphic design, photo compositing, and video making. We focused on artists doing digital art rather than physical art, because our goal is to recommend these videos in the context of creative software. 

In this section we provide an overview of creative live-streams based on our exploration and analysis, as well as formative findings from a survey with viewers of one live-stream channel to understand their motivations for watching. Our analysis supports our hypothesis that live-streamed videos are a viable source for creative inspiration, further motivating our goal to make them available to users in the context of their own workflows.

\addtocounter{footnote}{-4}
\stepcounter{footnote}\footnotetext{\url{www.twitch.tv/videos/154575884}}
\stepcounter{footnote}\footnotetext{\url{www.youtube.com/watch?v=RtswnAYbrdk}}
\stepcounter{footnote}\footnotetext{\url{www.youtube.com/watch?v=jP5fKeG1CkU}}
\stepcounter{footnote}\footnotetext{\url{www.twitch.tv/videos/152518965}}

\subsection{What are creative live-streams?}
There are two main things that make live-streamed videos different from other types of videos demonstrating creative work, such as tutorials. First, live-streamed videos usually show a real-world process, not a contrived or planned one like a tutorial might. Second, artists often start without an exact goal in mind, and it can be inspiring to watch someone go from a blank canvas to a beautiful piece of work, seeing the decisions and mistakes they make along the way.

Videos typically show a screencast of the artist's full computer screen and a webcam view of the artist's face (\autoref{fig:liveclips_streamers}). Artists spend most of their time in the creative software they are using, occasionally switching to a browser to find content or to a different application to export or view something. Some artists use screencasting software that overlays keyboard shortcuts and/or the names of tools when they are used, but this is relatively rare.

Artists often narrate while they work. Live-streams also feature a live chat, allowing viewers to communicate with each other and the artist. Artists sometimes read chat messages and respond to them verbally. Some live-streamed videos feature a host as well as an artist, in which case the host will read questions from the chat to the artist. Creative live-streams tend to have very tight-knit communities; viewers return daily or weekly, and fellow artists support and promote each other's streams.

\subsection{Survey: Why do people watch creative live-streams?}
To understand the motivations behind creative live-stream viewers, we conducted an anonymous survey with viewers of Adobe's creative channel on YouTube. Adobe is known for their large variety of creative software, and this channel features four different artists for three days every two weeks. The survey was posted periodically in the live chat, and was online for one month, capturing viewers of 12 different artists. The survey included questions about viewers' experience with creative software, the reasons they watch creative live-streams, and where they watch them (Twitch, YouTube, etc.). The survey took around five minutes to complete.

In total, 86 people responded to the survey. 84\% of participants had watched live-streams on this channel before, with 55\% saying they watch them regularly. 50\% of participants said they also watch creative live-streams from other sources, including other YouTube channels, Twitch, and Facebook. All participants had at least some experience using creative software.

\subsubsection{Viewers watch creative live-streams for inspiration, learning, and community}
Participants were asked to select all applicable reasons they watch creative live-streams from a list of options. The top two reasons selected were ``to learn how to be a better artist'' (86\%) and ``to get inspired'' (84\%). Other popular answers included ``to learn how to use creative software'' (74\%) and ``to be a part of the community'' (65\%). This suggests that creative live-streams can be useful content for both inspiration and learning.

In free-form text, 21 participants specifically mentioned inspiration as a main motivation for watching. Several participants (9) also mentioned that the videos helped increase their own motivation and confidence as artists. As one participant explained, ``[I] like watching artists work because it takes the mystery out of what they do''. Many participants (20) also mentioned how watching live-streams affects their own work; e.g., they learn techniques they can apply to their own workflow and get inspired with new ideas.