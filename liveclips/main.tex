
% Load basic packages
% \usepackage{balance}       % to better equalize the last page
% \usepackage{graphics}      % for EPS, load graphicx instead 
% \usepackage[T1]{fontenc}   % for umlauts and other diaeresis
% \usepackage{txfonts}
% \usepackage{mathptmx}
% \usepackage[pdflang={en-US},pdftex]{hyperref}
% \usepackage{color}
% \usepackage{booktabs}
% \usepackage{textcomp}
% \usepackage[sort,nocompress]{cite}
% \usepackage{perpage}
% \MakePerPage{footnote}
% \usepackage{multirow}

% % Some optional stuff you might like/need.
% \usepackage{microtype}        % Improved Tracking and Kerning
% % \usepackage[all]{hypcap}    % Fixes bug in hyperref caption linking
% \usepackage{ccicons}          % Cite your images correctly!
% % \usepackage[utf8]{inputenc} % for a UTF8 editor only

% % If you want to use todo notes, marginpars etc. during creation of
% % your draft document, you have to enable the "chi_draft" option for
% % the document class. To do this, change the very first line to:
% % "\documentclass[chi_draft]{sigchi}". You can then place todo notes
% % by using the "\todo{...}"  command. Make sure to disable the draft
% % option again before submitting your final document.
% \usepackage{todonotes}

\chapter{LiveClips: Contextual recommendation of inspirational clips from live-streamed videos}

\begin{figure}[b!]
\centering
  \includegraphics[width=\textwidth]{liveclips/figures/all_interfaces.png}
  \caption{Three interface prototypes demonstrate how video clips taken from creative live-streams can be embedded in a software tool as inspirational examples, implemented here in Adobe Photoshop. a) Contextual tooltips: When hovering over a tool icon, clips of that tool being used are shown. b) On-demand search: Pressing Photoshop's search button or Ctrl-F brings up an extended search interface with task-level clip examples. c) Ambient side panel: an always-visible panel updates periodically with clip examples based on the user's recent tool use. }~\label{fig:liveclips_photoshop}
\end{figure}


Getting inspiration through examples is a key part of the creative process. While one popular source for examples of creative work is online communities like Behance and Dribbble, there is a growing interest in seeing the process behind artists' work, not just the final product. An increasingly popular way to share one's creative process is live-streaming. Live-streamed videos are a rich and rapidly growing source of inspirational content, but their length and volume makes them hard to search or browse. We introduce RePlay: a system for automatically selecting interesting segments from live-streamed videos and recommending them to users in the context of their creative workflow, providing access to relevant examples in the moment. We present three methods for recommending and displaying video clips with varying levels of contextual support, and implement them in a popular creative application, Adobe Photoshop. We compare the accuracy of RePlay's clip ranking to human ranking and present initial user feedback on the three prototypes.


\section{Introduction}
Browsing and exploring inspiring examples is a key part of the creative process \cite{Shneiderman2007, Shneiderman2002, Greene2002, Herring2009, Bawden1986}. Prior work has shown that seeing examples throughout the entire process, including the beginning, middle, and even towards the end, is valuable~\cite{Kulkarni2014, Siangliulue2015}. One popular source for creative examples is online communities such as 500px, Behance and Dribbble\footnote{\href{https://500px.com}{\nolinkurl{500px.com}}, \href{https://behance.net}{\nolinkurl{behance.net}}, \href{https://dribbble.com}{\nolinkurl{dribbble.com}}}. However, these tend to showcase finished projects, which give the viewers little to no insight about \textit{how} or \textit{why} a project was created. 

Recent work has shown that seeing the process behind an artist's work is beneficial for creativity, as it encourages self-reflection on one's own process and methods \cite{Kim2017}. Some artists share works-in-progress, how-to tutorials, and videos describing the process that leads them to a final product, but these highly curated windows into process require time and effort for the creators to produce and share. 

Many artists have begun to broadcast live video as they work on tasks such as graphic design, crafting, drawing, and music through platforms like Twitch and YouTube\footnote{\href{www.twitch.tv}{\nolinkurl{twitch.tv}}, \href{www.youtube.com}{\nolinkurl{youtube.com}}} (\autoref{fig:livestream_examples}). Live streaming allows creators to share their unedited process \emph{while} they work. As a result, there is a rapidly growing collection of archived live stream videos that contain both inspirational and educational content regarding artists' creative processes. However, finding moments that are relevant and personally inspiring to a creator from this large collection of videos can be difficult, making live streams an under-utilized source for potential examples.

\begin{figure}[t!]
\centering
  \includegraphics[width=\textwidth]{liveclips/figures/examples-horizontal.png}
  \caption{Examples of creative live streams on Twitch, YouTube, and Facebook. Artists stream videos of themselves working on creative projects. Sources for video screenshots, from left to right: \href{http://bit.ly/2SK5zWE}{\nolinkurl{bit.ly/2SK5zWE}}, \href{http://bit.ly/2Bv9Y69}{\nolinkurl{bit.ly/2Bv9Y69}}, \href{http://bit.ly/2SJYFRa}{\nolinkurl{bit.ly/2SJYFRa}}, \href{http://bit.ly/2TK12Rq}{\nolinkurl{bit.ly/2TK12Rq}}}~\label{fig:livestream_examples}
\end{figure}

This chapter introduces LiveClips, a system for automatically selecting inspirational segments from long live streamed videos and recommending them as examples to users in the context of their creative workflow. We choose to present examples inside the user's software in an effort to make examples more available throughout the entire creative process.
%based on the known benefits of contextual in-application learning \cite{Grossman2010a, Pongnumkul2011, Kelleher2005, Dontcheva2014}. 
Contextually available examples increase the likelihood of unexpected, or ``serendipitous'' discoveries, which research has shown can spark new ideas in a wide range of creative domains, such as scientific research, writing, and visual art \cite{Bawden1986, Benjamin2014, Foster2003, Erdelez1999}. LiveClips' goal is to make examples pervasive in the creative process, to promote and encourage serendipitous moments of inspiration. 

LiveClips combines telemetry and computer vision techniques to automatically segment long videos into short 25-second clips, crop clips intelligently to a thumbnail size for easy viewing, and recommend clips to creative software users based on their usage behaviour. We demonstrate LiveClips' method by using it to automatically extract and rank clips from 17 live streamed videos of artists working in two popular creative applications, Adobe Photoshop and Illustrator. We focus on digital art tasks such as design and illustration, as these are currently popular tasks for live streaming, and the software used for these tasks has wide audiences.

We compare LiveClips' ranking algorithm to human ranking and find that LiveClips is able to predict a clip's inspirational value with reasonable accuracy. To demonstrate how short inspirational clips can be contextually embedded in creative software, we present a design space and three prototypes that lie within this space, implemented in Adobe Photoshop (\autoref{fig:liveclips_photoshop}). Initial user feedback suggests that this approach is promising and warrants further study. In summary, we make the following contributions:

\begin{itemize}
\item a formative understanding of creative live streaming from the perspective of both streamers and viewers,
\item a method for extracting short (25-second) clips from long live streamed videos and cropping them to thumbnail size,
\item an approach for selecting example clips to present inside creative software based on their visual properties, the user's tool usage, and the presentation location within the software,
\item three prototype implementations in a creative application and validation of the ranking approach with human raters. 
\end{itemize}

\section{Related Work}

\subsection{Creative Live Streaming}
Perhaps the three most popular genres for live streams are video gaming \cite{Pellicone2017, Lessel2017, Sjoblom2017, Hamilton2014}, programming \cite{Faas2018, Haaranen2017}, and lifestyle \cite{Lu2018a, Tang2016}. Popular live streaming platforms that host a variety of live stream genres include Twitch, YouTube, Facebook, Instagram, and Periscope. Creative live streams can be found on all of these platforms, but platforms dedicated specifically to \textit{creative} live streaming have also emerged, such as Picarto, Pixiv Sketch, and Behance \footnote{\href{www.picarto.tv}{\nolinkurl{picarto.tv}}, \href{https://sketch.pixiv.net/lives}{\nolinkurl{sketch.pixiv.net/lives}}, \href{https://behance.net/live}{\nolinkurl{behance.net/live}}}.

Live streaming democratizes the studio-apprentice model, enabling anyone to see experts' in-context choices by working alongside them \cite{Schon1985}. Section \ref{sec:liveclips_formative} of this chapter describes the motivations and challenges of both streamers and viewers of creative live streams, comparing and contrasting them with prior work on live streaming in other domains. We use Twitch's definition of creative work: ``visual art, woodworking, costume creation, prop building, music composition, or any other process in which you entertain and connect around a creative activity'' \cite{Moorier2015}. These activities focus on creating a novel artifact, unlike typical video games or lifestyle streaming.

\subsection{Examples are Important for Creative Inspiration}
Searching and browsing examples is an important part of the creative process \cite{Shneiderman2002, Shneiderman2007, Greene2002, Herring2009, Muller-Wienbergen2011, Bawden1986}. 
While search engines are a common tool for finding examples, directed search can prevent serendipitous discovery \cite{Benjamin2014}. Happening upon an unexpected or even seemingly unrelated example can spark new ideas that the creator wouldn't have otherwise thought of \cite{Erdelez1999, Benjamin2014}.
Creative software can support serendipitous discovery by presenting the user with examples while they work \cite{Bawden1986, Kulkarni2014, Herring2009}.

The selection of examples is important; prior work shows that diverse and far-ranging examples lead to more novelty in creative output \cite{Chan2011} and more diverse sets of ideas \cite{Siangliulue2015a} than examples that are similar to each other and the task at hand. The timing of examples is also important; Lewis \textit{et al.} \cite{Lewis2011} show that people can be primed to be more creative through exposure to examples before a creative task, while Kulkarni \textit{et al.} \cite{Kulkarni2014} show that seeing examples early on in the process as well as interspersed throughout the process improves creativity. However, diverse examples can actually be harmful if they are presented while the user is being productive \cite{Chan2017}. Siangliulue \textit{et al.} \cite{Siangliulue2015} found that the most novel ideas occur when examples are only shown at the user's request rather than automatically shown by the system. However, users may not always remember to look for examples, and so systems that ambiently update or recommend examples when the user is idle have also shown creative benefits \cite{Siangliulue2015, Rhodes1996}.

LiveClips selects a diverse set of examples that are relevant to the user's context. Guided by the research above, our prototype implementations explore variations in the timing and availability of examples.

%\subsection{Inspirational vs. educational content}
%In the context of creative work with software (\textit{e.g.}, design, digital painting), educational content consists of instructions for achieving certain tasks or techniques, learning material for how to use tools in the software, and tips or advice for creating good work (\textit{e.g.}, \cite{Grossman2010a, Pongnumkul2011, Kelleher2005, Chi2012}). Inspiration on the other hand can come from almost anywhere \cite{Cobbledick}, but a prominent type of content for inspiration in research on creativity is examples of others' work \cite{Muller-Wienbergen2011, Shneiderman2002, Herring2009, Kulkarni2014, Siangliulue2015a, Siangliulue2015}. One key difference therefore between educational and inspirational content is that educational content focuses more on the \textit{technique} behind accomplishing something, whereas inspirational content focuses more on the \textit{content} being created. There can be overlap between these categories; creative live streams are one example of content that can be both inspirational and educational. Their main focus tends to be on the content the artist is creating, but along the way the artist may also demonstrate specific techniques or mention helpful tips. In this way creative live streams bridge the gap between learning and inspiration. The next section in this paper discusses the content of creative live streams in more detail.

\subsection{Contextual Recommendations Support Learning in Software}
Given the increased focus in education (\textit{e.g.}, \cite{Prince2004}) and software learning (\textit{e.g.}, \cite{Greene2002, Grossman2010a}) on active learning, or ``learning while doing'', we believe similar benefits may arise by integrating the process of inspiration with the process of doing. Despite the known benefits of seeing examples throughout the creative process \cite{Kulkarni2014}, most creative software tools today lack support for in-context inspiration. Contextual presentation of learning content is an effective method for supporting learning while doing \cite{Grossman2010a, Matejka2011, Ichinco2017, Matejka2009}; in this work we explore whether contextual presentation of examples can similarly support ``inspiration while doing''. 

Embedding any kind of content in-app runs the risk of interrupting or distracting the user. Therefore, in-app content should be unobtrusive but easy to access \cite{Grossman2010a}. Tooltips are one promising avenue for this, as they only require a hover to access and can be easily dismissed by mousing away. Inspired by ToolClips \cite{Grossman2010a}, this work also uses tooltips as one potential interface for contextual assistance. 
%It is also beneficial to include multiple different examples of tool or command use to help users better understand how it works outside of any one particular context \cite{Grossman2010a, Lafreniere2014, Ichinco2017}. 
CommunityCommands, a command recommender for AutoCAD \cite{Matejka2009}, demonstrated that personalizing recommendations based on the user's own tool use makes them more likely to be helpful. Over a 6-week user study of CommunityCommands, Li \textit{et al.} \cite{Li2011} found that recommendations based on the user's short-term tool use are preferred over those based on the user's all-time tool use, as the former tend to be more contextually relevant. LiveClips similarly bases recommendations on the user's recent tool use.

% \begin{figure}[b!]
% \centering
%   \includegraphics[width=\columnwidth]{liveclips/figures/streamers.png}
%   \caption{Examples of creative live streams on Twitch and YouTube. Artists stream videos of themselves working on creative projects such as graphic design, illustration, and photo editing. (Sources for video screenshots: top left\protect\footnotemark, top right\protect\footnotemark, bottom left\protect\footnotemark, bottom right\protect\footnotemark)}~\label{fig:liveclips_streamers}
% \end{figure}

\subsection{Segmenting Videos to Make Them More Browsable}
This work uses videos as the source for inspirational examples. Videos have the advantage over text and static images that they show the process of using a tool, rather than just the outcome, and they provide visual demonstrations that are directly relatable to the visual user interface \cite{Grossman2010a}.

Extensive prior work has explored automatically generating educational video clips from screencast videos of software use \cite{Pongnumkul2011, Chi2012, Banovic2012, Lafreniere2014, Nguyen2015}. Though some methods rely on matching telemetry data for these videos \cite{Grossman2010, Lafreniere2014, Chi2012}, others have shown that computer vision alone can be used to detect tool selection events when such data is not available \cite{Pongnumkul2011, Banovic2012}. In this work, we use telemetry data to segment live streamed videos into clips and rely on computer vision for the presentation and ranking of clips. 

Given a long video with associated tool data, prior work suggests techniques for extracting clips that demonstrate particular tools \cite{Pongnumkul2011, Chi2012, Lafreniere2014}, and guidelines for selecting clips with the best learning value \cite{Lafreniere2014}. Lafreniere \textit{et al.}'s recommended guidelines \cite{Lafreniere2014} include keeping clips short (15-25 seconds), using clips that show clear visual change to the document, and avoiding clips that show multiple unrelated actions. LiveClips builds on this approach to extract and rank clips from live streamed videos and explores how the characteristics of an inspiring clip might differ from that of an instructional one.

\section{Understanding creative live-streams}
Creative live-streams are a unique and rapidly growing source of data that have yet to be deeply studied. While some work has examined live-streaming in the context of video gaming (e.g., \cite{Pan2016, Hamilton2014, Sjoblom2017}), to our knowledge there is no work exploring the growing trend of live-streaming creative projects. To understand the current landscape of creative live-streams as well as their applicability to inspiration in software, we explored and watched videos on two popular platforms, Twitch and Youtube. We analyzed videos by 47 different artists working on creative projects such as digital illustration, graphic design, photo compositing, and video making. We focused on artists doing digital art rather than physical art, because our goal is to recommend these videos in the context of creative software. 

In this section we provide an overview of creative live-streams based on our exploration and analysis, as well as formative findings from a survey with viewers of one live-stream channel to understand their motivations for watching. Our analysis supports our hypothesis that live-streamed videos are a viable source for creative inspiration, further motivating our goal to make them available to users in the context of their own workflows.

\addtocounter{footnote}{-4}
\stepcounter{footnote}\footnotetext{\url{www.twitch.tv/videos/154575884}}
\stepcounter{footnote}\footnotetext{\url{www.youtube.com/watch?v=RtswnAYbrdk}}
\stepcounter{footnote}\footnotetext{\url{www.youtube.com/watch?v=jP5fKeG1CkU}}
\stepcounter{footnote}\footnotetext{\url{www.twitch.tv/videos/152518965}}

\subsection{What are creative live-streams?}
There are two main things that make live-streamed videos different from other types of videos demonstrating creative work, such as tutorials. First, live-streamed videos usually show a real-world process, not a contrived or planned one like a tutorial might. Second, artists often start without an exact goal in mind, and it can be inspiring to watch someone go from a blank canvas to a beautiful piece of work, seeing the decisions and mistakes they make along the way.

Videos typically show a screencast of the artist's full computer screen and a webcam view of the artist's face (\autoref{fig:liveclips_streamers}). Artists spend most of their time in the creative software they are using, occasionally switching to a browser to find content or to a different application to export or view something. Some artists use screencasting software that overlays keyboard shortcuts and/or the names of tools when they are used, but this is relatively rare.

Artists often narrate while they work. Live-streams also feature a live chat, allowing viewers to communicate with each other and the artist. Artists sometimes read chat messages and respond to them verbally. Some live-streamed videos feature a host as well as an artist, in which case the host will read questions from the chat to the artist. Creative live-streams tend to have very tight-knit communities; viewers return daily or weekly, and fellow artists support and promote each other's streams.

\subsection{Survey: Why do people watch creative live-streams?}
To understand the motivations behind creative live-stream viewers, we conducted an anonymous survey with viewers of Adobe's creative channel on YouTube. Adobe is known for their large variety of creative software, and this channel features four different artists for three days every two weeks. The survey was posted periodically in the live chat, and was online for one month, capturing viewers of 12 different artists. The survey included questions about viewers' experience with creative software, the reasons they watch creative live-streams, and where they watch them (Twitch, YouTube, etc.). The survey took around five minutes to complete.

In total, 86 people responded to the survey. 84\% of participants had watched live-streams on this channel before, with 55\% saying they watch them regularly. 50\% of participants said they also watch creative live-streams from other sources, including other YouTube channels, Twitch, and Facebook. All participants had at least some experience using creative software.

\subsubsection{Viewers watch creative live-streams for inspiration, learning, and community}
Participants were asked to select all applicable reasons they watch creative live-streams from a list of options. The top two reasons selected were ``to learn how to be a better artist'' (86\%) and ``to get inspired'' (84\%). Other popular answers included ``to learn how to use creative software'' (74\%) and ``to be a part of the community'' (65\%). This suggests that creative live-streams can be useful content for both inspiration and learning.

In free-form text, 21 participants specifically mentioned inspiration as a main motivation for watching. Several participants (9) also mentioned that the videos helped increase their own motivation and confidence as artists. As one participant explained, ``[I] like watching artists work because it takes the mystery out of what they do''. Many participants (20) also mentioned how watching live-streams affects their own work; e.g., they learn techniques they can apply to their own workflow and get inspired with new ideas.
\section{Design Space for In-Application Video Examples}
\label{sec:liveclips_designspace}
To help inform our decisions in developing LiveClips and explore how users might interact with such a system during their creative workflow, we outlined a design space for systems that present contextual video examples (\autoref{fig:liveclips_designspace}). This design space is informed by our formative findings as well as prior work on contextual assistance in software. The following three sections present the LiveClips system, which includes three alternative interfaces that explore three different points in this space. This section also highlights where RePlay and ReMap (Chapters \ref{chapter:replay}-\ref{chapter:remap}) fit in the design space as compared to LiveClips. Most notably, LiveClips differs from RePlay and ReMap in that it presents inspirational content rather than educational, and it presents short tool-focused clips rather than entire task-focused videos.

\begin{figure}[b!]
\centering
  \includegraphics[width=\columnwidth]{liveclips/figures/designspace.png}
  \caption{The design space for in-application video examples, as well as where LiveClips and RePlay/ReMap fit along each axis. LiveClips' prototype interfaces vary along the bottom four axes. }~\label{fig:liveclips_designspace}
\end{figure}

\subsection{Goal of Video Content}
Most prior work regarding contextual videos, including RePlay and ReMap, has focused on content that aims to educate the user, such as tutorials \cite{Pongnumkul2011} and helpful tips \cite{Grossman2010a}. Inspired by the known benefits of examples for inspiration \cite{Kulkarni2014} and our survey findings suggesting that live streamed videos are used for inspiration, this work explores how contextual live streamed videos might inspire users. These videos also have educational value, but this work focuses on them as a source of inspiration.

\subsection{Demonstration Level}
Videos of software use can be tool-focused or task-focused. A tool-focused video demonstrates using a single tool, while a task-focused video shows an entire task from beginning to end.  Tool-focused videos are useful for quick contextual help that does not take away from the user's current work, for example reminding the user how a tool works \cite{Grossman2010a} or showing alternative uses for a tool.
Task-focused videos tend to be longer and are less useful in-the-moment, as they show a specific task that may or may not be relevant. Live streams tend to be task-focused, as they usually follow an artist through an entire task; this work focuses on extracting shorter, more generalizable tool-focused examples from task-focused live streams. RePlay and ReMap also extract relevant moments from within longer task-focused videos, but unlike LiveClips, they are meant for users who proactively seek help with a particular task, so they allow the user to browse the entire video and follow along with it.

%The likelihood that this task will be directly relevant to the user is low \cite{?}. 
%In this work, we focus on creating short tool-focused clips out of longer task-focused videos based on the assumption that shorter bite-sized clips are more useful as in-context recommendation, as they do not take the user away from their current work like a long task-focused clip would. 
%In addition, videos of artists using software often highlight alternative uses for tools that viewers may not have seen before, and so seeing a variety of ways a tool can be used may provide both inspirational and educational value. 

%Many of today's popular creative software applications support a wide variety of workflows. As a result, many of their tools can be used for multiple different purposes and in different contexts. Even expert software users may not be aware of all the different ways a tool can be used. Thus we expect that in addition to providing inspiration, tool-based clip recommendations may help increase users' awareness of the different ways tools can be used. By narrowing in on a focused aspect of a video, we also provide this content in a way that does not detract too much from the task at hand, unless the user wants to see more.

\subsection{Video Cropping}
In-application examples are constrained by space; if content takes up too much space it becomes obtrusive, for example by blocking the canvas where the user is currently working. Therefore, contextual videos must be displayed at a relatively small size. For videos that were created using full-screen video capture (as most live streams are), this can be problematic. Leaving a video un-cropped allows the user to see the full context of interaction but makes it difficult to discern any detail. Since RePlay and ReMap left videos un-cropped, they allowed users to open a single video in a larger window when seeing more detail was necessary. While this may have been practical for task-focused learning where users have other cues to indicate a video's relevance before selecting one (\textit{e.g.}, caption previews), LiveClips aims to show users many examples at once of tool use that may only take place in a small part of the screen. Prior work has explored how to best present video demonstration clips when they cannot be viewed in full screen, by cropping or enlarging sections in the video where mouse movement and canvas changes occur \cite{Chi2012}. Inspired by this, LiveClips crops videos to the area that shows the most visual change, in an effort to make examples focused and easy to browse.

\subsection{Number of Video Clips Shown}
Examples could be displayed one at a time, taking up the least amount of space but also providing less variety. Alternatively, displaying many videos gives users more options but potentially overwhelms them. Prior research on contextual videos recommends displaying multiple videos to demonstrate the range of uses for a tool and increase the likelihood that the user will find at least one useful \cite{Lafreniere2014, Matejka2011}. Similar to RePlay and ReMap (which displayed five videos at a time), our prototypes display four videos at a time in an effort to provide some variety without overwhelming the user or taking up too much space.

\subsection{Visibility}
Examples can be always visible while the user is working (like with RePlay and ReMap), or visible only on demand. Examples that are always visible are more likely to be seen but can be distracting, while examples that are not easily visible and shown only on request are more likely to be ignored or missed \cite{Rhodes1996}. Different users may have different preferences for how visible they want their examples to be, just as some viewers like to watch creative live streams while they work, while others may not. Our prototypes explore three variations along this axis.

\subsection{Update Frequency}
Recommended examples can update in response to an explicit user action (\textit{e.g.}, a query), an implicit user action (\textit{e.g.}, being idle for some time or opening a new document), or automatically at a regular time interval. Updating in response to explicit user actions was appropriate for RePlay and Remap, which are intended to be used when users have a specific question. But for a system like LiveClips that aims to elicit inspiration, the ideal approach is less clear. The trade-offs between these strategies have been widely explored in the literature, and it seems there is no globally optimal solution \cite{Rhodes1996, Chan2017, Siangliulue2015}. Automatic updates can be useful because they require zero effort from the user and can highlight an example in the moment \cite{Rhodes1996}, but they can also distract the user during periods of focused work \cite{Chan2017} or make the interface feel uncontrollable \cite{ODonovan2015}. Updates in response to an explicit request are less distracting and give the user more control over their attention but rely on the user to know when an update would be useful \cite{Rhodes1996, Siangliulue2015}. Update frequency is also tied to Visibility; recommendations that are only visible on-demand only update (visibly) when the user requests them. As with Visibility, the ideal solution may depend on the user's preferences; our prototypes explore three variations.

\subsection{Video Viewability}
When presenting non-static content such as videos in-app, the way in which the user interacts with and views the content may affect its usefulness. On the passive end, videos could play automatically when the content appears, which removes the need for the user to decide whether or not to watch a video but runs the risk of being distracting. On the other end, videos could be shown as a static thumbnail that only animates when the user intentionally interacts. Most prior work  \cite{Grossman2010a, Chi2012}, including RePlay and ReMap, has adopted the latter approach, however such work also suggests that in-context content should have a low threshold for engagement; the user shouldn't have to break too much from their task to interact with recommended content \cite{Grossman2010a}. LiveClips explores three variations along this axis.

\subsection{Context Level}
Examples can vary in how contextual they are to the user's behaviour. Low-level examples respond to the user's individual actions, such as the tools they use. High-level examples respond to the user's task or intent, which can be inferred from sequences of commands or by analyzing the document being edited. RePlay and ReMap use low-level context to augment queries, but the queries themselves can be either low- or high-level depending on the user. LiveClips mostly focuses on lower-level examples, but one of our three prototypes explores a potential way to measure task-level similarity between the user's actions and the examples clip's source video.
\section{RePlay System Overview}
RePlay takes as input a live-streamed video and telemetry data for the software usage in the video. It extracts short clips showing bite-sized chunks of the artistic process, crops clips to thumbnail size, and ranks clips for recommendation based on their properties as well as the user's context. In the current implementation, audio is removed to focus solely on the visual component of the videos. %(SYSTEM DIAGRAM)

Our approach is tool-centric: each clip focuses on one tool and RePlay recommends clips based on the user's tool use in the application. We focus on tools as a starting point, because prior work has demonstrated that short tool-focused clips can be effective for contextual learning~\cite{Grossman2010a} and creative software tasks are often centered around the use of different tools.

\section{User Interfaces}
Based on the design space outlined above, we present three alternative methods for displaying and recommending the video clips generated by RePlay in creative software (\autoref{fig:liveclips_photoshop}).
All three methods provide a link to the original video source underneath each clip, so that users can easily access it if desired. Each interface was implemented as a prototype HTML/Javascript extension to Adobe Photoshop.

\subsection{On-demand search}
Many software tools provide in-application search, which allows users to search for application features, help, and/or recent documents. We augmented Photoshop's in-application search interface to present four clips when the search window is opened (\autoref{fig:liveclips_photoshop}b). Clips are selected based on the user's tool usage during the entire work session. Users can also search the entire library of video clips from this window. We expect this interface to be most useful in moments where the user is stuck and wants new ideas. Each video is displayed as a thumbnail showing the first frame and plays when clicked on.

This interface has low visibility, and updates only in response to an explicit user action. It uses higher-level context to recommend examples. Playback requires intentional mouse clicks.

\subsection{Ambient side panel}
Since creative applications tend to include many panels, a natural location for examples is in a side panel. This interface explores examples that update ambiently as the user works (\autoref{fig:liveclips_photoshop}c). To avoid distracting the user during periods of focused work \cite{Chan2017}, the examples only update after the user has been idle for 10 seconds, indicating that they might be taking a break or trying to think of new ideas \cite{Siangliulue2015}. The panel shows four clips based on the user's recent tool use. Mousing over a clip plays it, providing a low threshold for interaction without distracting the user by playing video clips automatically.

This interface has high visibility, updates in response to implicit user actions, and uses mid-level context to recommend examples. Playback requires mouse movement without clicks.

\subsection{Contextual tooltips}
For an interface that lies between an on-demand window and an always-visible panel, our third approach shows examples in tooltips that appear when the user hovers over a tool icon for 3 seconds (\autoref{fig:liveclips_photoshop}a). Inspired by ToolClips \cite{Grossman2010a}, we believe tooltips may be a useful location for video examples. They can be activated by the user with minimal effort, are easily dismissed, and will not display at all if the user is working quickly or activating tools with keyboard shortcuts.
Tooltips are a natural place for tool-centric examples; the video clips shown are based on the tool selected. Our implementation shows four clips when the user hovers over a tool, and clips play automatically when the tooltip appears.

This interface has mid-level visibility, updates on demand, uses low-level context (tools) to recommend examples, and requires no interaction for video playback. 

\section{RePlay system for generating clips}
The RePlay system comprises three stages: 1) extracting short 25-second clips based on tool use from long videos, 2) cropping clips so that they can be displayed at thumbnail size, and 3) ranking clips for the three interfaces described above. 

Our approach uses a hybrid of telemetry (recorded usage data) and computer vision to understand and analyze the videos. We note that this could be accomplished entirely with telemetry (as in \cite{Grossman2010, Lafreniere2014}) by instrumenting the artists' software with detailed logging. Conversely, it could be accomplished entirely with computer vision, as prior work has shown that mouse movement and tool selection can be detected from videos alone \cite{Banovic2012, Pongnumkul2011}. In our case (which is not uncommon), we have some usage data that is not fully detailed. Our approach aims to be a catch-all that allows working with a mixed or inconsistent set of data. RePlay removes all audio from clips during processing, focusing only on the visual component.

\subsection{Available data}
To develop and test RePlay, we collected a sample dataset of live-streamed videos. In an effort to make the sample as representative as possible, we collected videos from two sources (Twitch and YouTube), from 17 different artists, and for two different software applications: Adobe Photoshop and Illustrator. Videos ranged in length from 30 minutes to 3 hours, giving us a total of 30 hours of video content from 17 different videos. 

For each video we had access to telemetry data at the level of tool usage, which includes time-stamped events for every selection and invocation of a tool (\autoref{fig:liveclips_usage}). Mouse clicks and canvas manipulation details were not available. Our dataset included the use of 36 different tools from the toolbars in Photoshop and Illustrator.  

\begin{figure}[b]
\centering
  \includegraphics[width=0.8\columnwidth]{liveclips/figures/usage.png}
  \caption{A sample excerpt of the time-stamped usage data we had for each live-stream video.}~\label{fig:liveclips_usage}
\end{figure}


\subsection{Extracting clips}
To extract short clips, we use a heuristic approach based on work by Lafreniere et al. \cite{Lafreniere2014}, who segment videos to create instructional clips of tool use. Our focus is on creating inspiring clips, which are different from instructional clips in that they require less attention to specific details (such as setting a tool's parameters) and more attention to the content being created. For the purposes of segmenting initial clips, our method is very similar to Lafreniere et al.'s method. We focus on inspirational value in the cropping and ranking stages.

Given a tool, our goal is to create clips showing its use. First, we group all consecutive invocations of the tool and include the tool selection event if it happened within 10 seconds of the first invocation. We add 2 seconds of padding to the beginning so that the viewer can see the tool being selected or invoked. We then trim clips to 25 seconds, as prior work has indicated that 15-25 seconds is a desirable length for in-app video clip recommendations \cite{Lafreniere2014}. We also ignore tool events that are for navigation (e.g., zoom, scroll, select) as these tend to happen frequently in visual software as part of other tasks. We shorten clips where the context likely changed before it was over, i.e. if a document is closed or opened. If a clip is shorter than 15 seconds, we remove it.

This heuristic approach sometimes fails. For example, live-streaming artists often work meticulously with one tool for long periods of time (e.g., a brush tool to create artwork), making incremental changes that over time create a more drastic and impressive change. The first 25 seconds of this may not be very inspiring, so we also explore an alternative method for creating clips that brings the focus away from the specifics of a tool's use, and toward the content being created: for instances where a tool is used consecutively for longer than 25 seconds, we extract the entire section where that tool is used, and speed it up to be 25 seconds long. We refer to these as timelapse clips.

In our sample set of 30 hours of video, the above two methods combined produced 1,727 clips, 484 of which were timelapses. This is comparable with Lafreniere et al. \cite{Lafreniere2014}, who generated approximately 2500 clips from 25.5 hours of footage. RePlay generated between 1 and 939 clips for each of the 36 tools in our telemetry dataset.

\subsection{Cropping clips}
To present examples within an application, they have to be easily viewable by the user. A main design challenge with contextual clip recommendations is the space constraint: prior research has shown that in-app video clips should be unobtrusive and not take up too much of the user's screen \cite{Grossman2010a}. Since live-streamed videos typically include the artist's entire screen, simply resizing them to a small thumbnail size will make most of the interesting detail hard to see. Existing methods for cropping videos to a good thumbnail size include cropping it to only the region of the document that changes in the clip \cite{Grossman2010} or cropping to only relevant UI regions \cite{Chi2012}. While more animated effects such as ``pan and zoom'' might be appealing for providing both context and detail, Chi et al. \cite{Chi2012} found that too much animation is disorienting for brief video clips.

Since prior work has shown that visible change is an important factor affecting the usefulness of a video clip \cite{Lafreniere2014}, RePlay attempts to crop each video clip to the area that changes most in the clip, so that the viewer's attention is drawn to the changes that are happening. For inspiration, we are mainly interested in visual change on the canvas. However, there are many other types of visual change that happen in these videos, such as switching windows, opening dialogs, and zooming in and out. Some of these changes (e.g., zooming) are simply uninteresting, and others (e.g., setting parameters in a dialog) may be helpful in a tutorial but are less interesting for short inspirational clips. 
 %The main goal for this content is to expose users to what's possible. Though users might want to see more specifics when they try and follow the artist's method, they will only reach this point if they find the content inspiring in the first place. In such situations, they can click go to the source video and watch the full original video on the Web. 
%
RePlay therefore excludes these types of change before identifying the area that exhibits the most visual change. This leaves canvas changes and other UI changes. Although we are primarily interested in canvas changes, we leave the task of separating these from UI changes to future work.

To filter out these ``uninspiring'' visual changes, RePlay uses computer vision to exclude visual change caused by navigational movement, changing application windows, and the artist moving in the webcam view. To detect navigation events, RePlay does simple feature detection and point feature matching in MATLAB across consecutive frames to identify and exclude frames where there is a significant amount of motion (likely due to panning or zooming). To detect application window changes, RePlay identifies moments where more than 90\% of the pixels change between two consecutive frames, and ends the clip before this change occurs. To detect change caused by the artist's webcam view, RePlay uses face detection in MATLAB to locate faces that are consistently present in a bottom corner of the screen (which is the customary place for streamers' webcam views) throughout the video, and mask out the corner containing those faces.

To determine the area of most change after filtering out uninspiring changes, RePlay first calculates the pixel-wise difference between each pair of consecutive frames in the video clip. It then computes the average difference over all of these difference frames. There are many changes around the edges of a clip where artists open menus and panels, but these are not very interesting. The interesting changes are on the canvas, which is typically in the middle of the screen. Thus, RePlay trims off 100 pixels from all four sides. Next, RePlay further trims off all sides where the pixel values in the average difference frame are less than a given threshold (which we set to 1/4 of the maximum value in the frame). This allows us to avoid specifying a desired crop size, since some clips may be already zoomed in on the part of the canvas being changed, requiring minimal cropping, whereas others may be zoomed out, requiring substantial cropping to highlight the changing area. \autoref{fig:liveclips_change} shows an example of the average difference frame from a clip and the crop that results from it.
.

\subsection{Ranking clips for recommendation}
The set of candidate clips generated is much too large to be useful, as only a few videos will eventually be shown to the user at any given time. As Lafreniere et al. \cite{Lafreniere2014} found, over 50\% of automatically selected clips are of poor quality, so choosing good clips from the candidate set is an important step. This section outlines the criteria RePlay uses to rank clips.

\subsubsection{Time in the original video}
Live-streamed videos are several hours long and often show an entire project happening from start to finish. The closer an artist is to the end of their project, the more finished content they are likely to have on their canvas, and thus the more inspiring it is likely to be% (figure showing comparison if time)
. For each clip, RePlay divides the start time of the clip by the total length of the video to obtain a number between 0 and 1 indicating how far along in the video the clip occurs. Larger numbers are better because they indicate that the clip occurs closer to the end.

\subsubsection{Amount of visual change}
 Lafreniere et al. \cite{Lafreniere2014} found that showing clear visual change in short clips was most closely correlated to user preference. To determine how much visual change a given clip shows, RePlay takes the cropped difference frame generated in the previous stage (\autoref{fig:liveclips_change}), and averages it across the $x$ and $y$ dimensions to obtain a numeric value representing the average amount of change in that clip (a larger number = more change).

\begin{figure}[b!]
\centering
  \includegraphics[width=\columnwidth]{liveclips/figures/change.png}
  \caption{An example of a clip's visual change averaged across time (with change caused by navigation, application window switching, and the artist's face moving removed). From the brightness values of the pixels, we can see that the artist opened some dialogs, and drew some lightning strokes. The green box shows how RePlay crops the clip to focus on the area that is changing. }~\label{fig:liveclips_change}
\end{figure}

\subsubsection{User context}
User context is the final factor for selecting which video clips to present to the user. This metric varies in each of our three prototypes, to explore different levels of user context. RePlay's current implementation uses tool use to measure context. RePlay first calculates a ``total'' score for each clip by normalizing the time and visual change scores from above to be between 0 and 1, then taking the average. (For simplicity, we weight the two metrics evenly.) 

\textbf{Low-level context:} The contextual tooltips interface uses low-level context to recommend examples; video clips are selected based on the tool that the user hovers over. For a given tool, RePlay orders all clips of that tool by total score, then picks the top four clips. RePlay requires that all examples come from different source videos, to ensure a variety of content. If some of the top examples are from the same original video, the system goes down the list until it has a set from four different source videos. %In practice, this rarely happens. 

\textbf{Mid-level context:} The ambient side panel uses mid-level context to recommend examples; video clips are selected based on the last four tools the user has used. For each of those tools, RePlay chooses the clip with the highest total score. If some of these clips are from the same original video, RePlay instead picks the next top clips that are from different videos, to ensure variety.

\textbf{High-level context:} The on-demand search interface uses high-level context to recommend examples; the user's entire session of tool use is taken into account as well as the entire video from which each clip was extracted. %Many tools can be used for a variety of tasks; for example the brush tool in Photoshop could be used for making a digital painting, or for brushing on a mask to edit a photograph. It is likely that the overall distribution of tools used differs for the above two tasks: a user making a painting would likely spend most of their time using the brush tool and eraser tool, whereas a user editing a photograph might spend some time with the brush tool, and other time with tools such as the clone stamp and the spot removal tool. To recommend videos that more closely match the user's overall task, we therefore look at the time distribution of tool use in each video and compare it with the user's distribution of tool use. 
For each full-length video, we store the overall percent of time spent using each tool, and compare this with the overall percent of time the user has spent using each tool in their current session. To measure the ``difference'' between a video and the user's session, we sum the absolute differences between percentages for each matching tool (each is a number between 0 and 1), and for every tool used in the video that the user has not used at all, we add 1 to the sum. The video with the smallest difference sum therefore represents the closest match to the user's task. RePlay chooses the live-stream videos with the four smallest difference sums. From each video it picks the  clip with the highest total score.% that shows a tool that the user has used in the session.

%To make recommendations RePlay recommendation engine compares all of the user's recent activity (e.g., which tools are used) to the overall activity in the livestream from which the the clip was extracted. 
%Inspired by CommunityCommands \cite{Li2011}, we consider only usage from the current session, which we define as the time since the user opened the application. For each video, we calculate the percent of time spent using each tool and compare this distribution with the percent of time the current user has spent using each tool. We select the top four videos whose distributions match most closely with the user's, and from each video we select the top ranked clip that shows a tool the user has used. 


%Second method: Recommendations are chosen based on the user's recent tool use, at a lower level than the search interface; every time the the recommendations update, it chooses the top ranked clip for each of the last four tools the user has used.
\section{Evaluation}
\label{sec:liveclips_eval}
To evaluate whether our ranking metrics are effective at finding inspiring clips, we conducted a study on Amazon Mechanical Turk. This study did not evaluate clips in the context of creative software. Instead, it focused on whether timing and visual change are good general predictors for inspiring content. 

As described in the System section, LiveClips generated 1,742 clips from our dataset of 30 hours of video of Photoshop and Illustrator use. We randomly sampled 129 of these clips to include in our evaluation set. To ensure we had coverage of all tools in our dataset, we randomly chose between 2 and 10 clips per tool. Since some tools are much more popular than others (\textit{e.g.}, Photoshop's brush tool had 939 clips whereas the burn tool only had 2 clips), this method allowed us to have a smaller sample while still representing every tool.

\begin{figure}[t!]
\centering
  \includegraphics[width=0.8\columnwidth]{liveclips/figures/mturk.png}
  \caption{An example of a pairwise comparison completed by Mechanical Turk workers. Each task involved five such pairs. In the introduction to the task, workers were asked to consider which 25-second video would inspire them to be more creative. }~\label{fig:liveclips_mturk}
\end{figure}

Workers compared video clips in pairs. Each Mechanical Turk task started with an overview page describing the task and showing an example pair of clips followed by five rounds of paired comparisons. Each comparison involved viewing two clips of the same tool and answering which clip they think would inspire them to be more creative (\autoref{fig:liveclips_mturk}). After completing all five comparisons, workers completed a short survey asking about their experience with Photoshop and Illustrator, and what types of creative tasks they do. Workers' results were only included if they completed all of the above steps. In an effort to ensure that they actually watched the videos, each comparison page only allowed workers to answer the question once both videos had played through once. Workers were paid \$1.50 per task, which took an average of 10 minutes and 20 seconds to complete. The same worker could do multiple tasks and would see five new pairs of videos (randomly selected) every time.

\subsection{Results}
In total, 481 workers completed 628 tasks, giving us a total of 3,140 paired comparisons. Most pairs were compared by 9 unique workers, though some ended up being compared more times. To balance results across clips, we considered only the first 9 comparisons for each pair. \autoref{table:liveclips_experience} shows the distribution of workers' prior experience with Photoshop, Illustrator, and various types of creative tasks. Notably, 73\% and 35\% of participants had at least some experience using Photoshop and Illustrator respectively.

\begin{table}[t!]
\small
\centering
\caption{The distribution of workers' experience with Photoshop and Illustrator, and the types of creative tasks they have experience with.}
\label{table:liveclips_experience}
\begin{tabular}{lll}
            &                           & \# Participants \\ \hline
Photoshop   & None                      & 130 (27\%)       \\
experience  & Beginner                  & 242 (50\%)      \\
            & Intermediate              & 91 (19\%)       \\
            & Expert                    & 18 (4\%)        \\ \hline
Illustrator & None                      & 311 (65\%)      \\
experience  & Beginner                  & 114 (24\%)      \\
            & Intermediate              & 46 (9\%)       \\
            & Expert                    & 10 (2\%)        \\ \hline
Creative    & Physical drawing/painting & 247 (51\%)      \\
experience  & Digital drawing/painting  & 147 (31\%)      \\
            & Photography               & 386 (80\%)      \\
            & Photo editing             & 290 (60\%)      \\
            & Design                    & 144 (30\%)     
\end{tabular}
\end{table}

Since only clips showing the same tool were compared, our analysis only examines ranking agreement within tool groups. Each tool group had between 2 and 10 video clips. For groups with only 2 clips, we determine the human ranking of these two clips by ranking whichever clip was chosen more often first, and the other second. For all other groups, we use the Bradley-Terry model as implemented in \cite{Maystre2015} to infer a ranking of clips within that group based on the paired comparisons. Within each group, we use Spearman's $\rho$ to compute the correlation between human rankings and LiveClips rankings. LiveClips rankings are based on the total score for each clip (time and visual change combined).

\textbf{Overall agreement between rankings ranges between very weak and moderate.} Since the groups have different sizes, we cannot compute one overall measure of correlation, as it is unclear what the null distribution would be. \autoref{table:liveclips_agreement} (top) shows the average $\rho$ values for each group size. Note that $p$-values are not appropriate for such small group sizes, however the $\rho$ values still accurately represent the correlation between rankings. \autoref{fig:liveclips_agreement} (left) shows an aggregate comparison of all clip rankings. 

\begin{table}[t!]
\small
\centering
\caption[Average Spearman $\rho$ correlation between human ranking and LiveClips ranking for all clips (top) and all clips with >65\% agreement among turkers (bottom), compared within each tool.]{Average Spearman $\rho$ correlation between human ranking and LiveClips ranking for all clips (top) and all clips with >65\% agreement among turkers (bottom), compared within each tool. ``\# Groups'' for column $i$ refers to the number of tools that have $i$ clips. Positive $\rho$ values indicate a positive correlation, with correlations above .2 considered weak, above .4 considered moderate, and above .6 considered strong.}
\label{table:liveclips_agreement}
\textbf{All clips (n = 115)}\\
\vspace{5pt}
\begin{tabular}{l|llllllll}
\# Clips & 2    & 3 & 4 & 5   & 6    & 7    & 9    & 10   \\ \hline
\# Groups           & 18   & 5 & 2 & 1   & 1    & 1    & 2    & 2    \\
Mean $\rho$  & 0.11 & 0.3 & -0.2 & 0.6 & 0.09 & 0.54 & 0.21 & 0.48
\end{tabular}\\
\vspace{5pt}
\textbf{Clips with >65\% agreement (n = 64)}\\
\vspace{5pt}
\begin{tabular}{l|llll}
\# Clips & 2    & 3   & 4    & 5   \\ \hline
\# Groups           & 18   & 3   & 3    & 1   \\
Mean $\rho$  & 0.33 & 1.0 & 0.73 & 1.0
\end{tabular}
\end{table}

\begin{figure}[t!]
\centering
  \includegraphics[width=0.65\columnwidth]{liveclips/figures/agreement.png}
  \caption{Human ranking vs. LiveClips ranking for all video clips (left), and all video clips with >65\% agreement among turkers (right). Note that there is more overall agreement between rankings in the subset on the right.}~\label{fig:liveclips_agreement}
\end{figure}


\textbf{Agreement between rankings for less-disputed clips ranges between weak and very strong.} The main goal of LiveClips' ranking is not to establish an overall ranking of \textit{all} clips, but rather to ensure that bad clips are discarded, and that only the best clips are shown to the user. Therefore, any clips that had a large amount of disagreement between workers are likely not among the best. Indeed, if we restrict our selection of video clips to only those that either won or lost over 65\% of all comparisons they were a part of, this set consists of video clips that likely have a more obvious objective value. \autoref{table:liveclips_agreement} (bottom) shows the average $\rho$ values for each group size when restricted to this smaller set of clips (in our sample set, 64/115). Overall we see stronger correlations. \autoref{fig:liveclips_agreement} (right) shows an aggregate comparison of all these clips' rankings; we see fewer large disagreements here than in \autoref{fig:liveclips_agreement} (left).

Timelapse clips did not perform significantly better or worse than standard clips. However, there were far fewer timelapse clips than standard ones in our dataset (19 vs. 96), due to the fact that longer sequences of a single tool use were less common overall than shorter sequences. A larger dataset is needed to determine whether or not timelapse videos may be preferred over standard ones.

\section{Early User Feedback}
The Mechanical Turk evaluation explored the viability of LiveClips' ranking algorithm outside the context of an application. To get some initial user feedback on placing inspiring examples \textit{inside} an application, we recruited two casual Photoshop users, both male, to work on projects of their choice. They used Photoshop with all three interfaces enabled at once (on-demand search, ambient side panel, and contextual tooltips) (\autoref{fig:liveclips_photoshop}), and gave feedback while they worked.

Overall feedback was positive. Both participants found seeing video examples in the application useful and said that they wanted to see how others use tools and set parameters. The two participants differed in how they wanted to see examples. One participant preferred the tooltips, explaining that they were a nice level in between the on-demand search (which was easy to forget about) and the ambient panel (which they found distracting). They liked that if you happen to hover on a tool for an extra moment (which could happen by accident), it pops up and serves as a reminder that these examples exist, without taking away too much from the process. The other participant preferred the panel, and wanted to be able to refresh recommendations on demand and see a large variety of content. Both participants clicked on the videos and wanted to watch them in the browser where they could see them larger. 

This early user feedback is encouraging and shows the opportunity around embedding inspirational examples in creative software.


\section{Discussion}

\subsection{Can We Really Predict What Will Be Inspiring?}
The results from the ranking evaluation presented in the previous section should be taken with a grain of salt. The workers who participated in this study were not all users of creative software, and the video clips were presented out of context. The evaluation was intended as an initial baseline to determine whether LiveClips' ranking algorithm can reasonably identify good clips. The subjective nature of the question workers were asked meant that we could not include a ``ground truth'' comparison to filter out lazy turkers. We did however enforce that the entire video clips played at least once before workers could select an answer.

Aside from the questions that using crowd-worker participants raise, the idea of ``inspiration'' is subjective and hard to predict. What one person finds inspiring another may not. Whether someone finds something inspiring could even change depending on the time at which they see it. Despite these challenges, we had reasonable agreement among workers overall; the average percent of agreement across all pairs of clips was 74\% (SD 14.5). Even if Mechanical Turk workers \textit{were} a 100\% reliable source, we would still expect some disagreement, due to the subjective nature of the question.

The LiveClips algorithm can reasonably identify clips that most workers agree are either good or bad. It is important to keep in mind that the main purpose for this algorithm is to address the ``cold start'' problem. Once people start using an interface with video examples, LiveClips could continually improve the ranking algorithm based on user behaviour and preferences, exhibited through how much people interact with the videos. %There are other low-level features of clips identified by Lafreniere \textit{et al.} \cite{Lafreniere2014} that could also be included in a ranking algorithm, such as whether a clip shows parameters being set, but we chose not to include these in the LiveClips algorithm because it is likely that the \textit{content} of a video's artwork will be the main factor that determines its inspirational value. but this is a hard problem.

\subsection{How Diverse Should the Set of Examples Be?}
Research on examples and creativity is rather divided regarding whether diverse examples that are distantly related to the user's task are better than a narrow set of examples that are more closely related to the user's task. Some work has shown that more diverse, far-off examples improve creativity by encouraging people to think more broadly and try new things \cite{Chan2011, Siangliulue2015a}, while other work has shown that similar examples are more useful as they are more relevant to the user \cite{Chan2015}. More recently, Benjamin \textit{et al.} \cite{Benjamin2014} propose letting the user decide by providing an adjustable slider that determines the diversity of recommended examples, based on the idea that the need for more diverse or more similar examples may differ depending on where the user is in their process.

Another option could be to let users search by example, a feature that is now common in search engines. Rather than having to specify an exact query, users could provide an example video and ask for ``more like this''. However even in this case the diversity of results is an important factor to consider.

%let's cut this for now.It's good but not as connected to inspiration as the the previous sections, which work well together
%\subsection{How much information should recommendations show?}
%Prior research has shown the benefits of being transparent with users about how recommendations are chosen \cite{}. In our three interfaces, contextual tooltips are transparent by nature, as they show only clips of the selected tool. However the method for choosing clips in the search window and panel is currently rather opaque. It may be helpful to provide users with more detail regarding why a clip was chosen, for example because they used a certain tool recently. 

%More generally, providing users with enough information scent to make a quick but informed decision about whether they should give their attention to an example is important. This highlights a challenging trade-off in the design of contextual interfaces: showing users enough information without making it too costly to review this information \cite{?}. While there are many additional things that could be shown along with a video clip (a description of it, the audio transcript, the tools used in it, an overview of the task it shows, etc.) and existing methods for highlighting important activity within the clip (highlighting mouse movement or areas of change \cite{}, showing what keys are being pressed when \cite{}), incorporating all of this information would quickly become overwhelming. Prior work has shown that a brief text description can help users efficiently scan a set of video clips \cite{}; this may be a good starting point.
\section{Limitations \& Future Work}
This chapter presented initial work exploring ways to bring inspiring examples into the creative process, and a new type of content from which to draw such examples: creative live stream videos. We provide suggestive results that our algorithm can select potentially inspiring clips, and some initial user feedback on our prototype interfaces indicating that this is a promising avenue for future work. Rather than conduct a rigorous controlled study (which is difficult and often inappropriate for evaluating goals like creativity and inspiration \cite{Shneiderman2007}), this chapter aims to introduce this space and lay out the possibilities for future work to explore. In this section, we discuss a few main directions for such work.

\subsection{More Robust Segmentation and Change Ranking}
Our current methods for detecting visual change use simple computer vision approaches, and exhibited some failure cases. We have yet to try more sophisticated deep learning methods to segment this data but this is a promising direction of future work that could help improve the classification of panning, zooming, and parameter setting. Having more robust and available usage data could also improve the detection of such actions. For example, Lafreniere \textit{et al.} \cite{Lafreniere2014} used instrumented software to gather both videos and detailed usage, which allowed them to calculate visual change directly by counting the number of pixels on the artist's document that change during a video clip. As of recently, streamers on Behance (\href{https://behance.net/live}{\nolinkurl{behance.net/live}}) can use a plugin that logs their tool use while they stream in creative software. This data can be used to segment live stream archives \cite{Fraser2020}, but such segmentations would still benefit from additional visual analysis to identify events that are not captured in usage logs and understand visual properties of the artist's work. Finally, to make use of the large amount of video content that already exists online with no available telemetry, a deep learning system could be trained on those videos that do have usage data, and then applied to videos that do not.

\subsection{Making Use of Audio and Chat Logs}
Live streamed videos come with additional data that this work did not make use of: audio in the form of the streamer's narration and responses to questions (which can be transcribed to text) and chat logs from viewers of the stream. Making use of audio narration in software tutorial videos is an open problem \cite{Chi2012} as narrations don't always align exactly with the artist's actions. Live streamed videos have a similar problem that is exacerbated by the fact that artists are not always narrating what they are doing; as our formative work found, sometimes streamers answer questions from the chat and other times they talk about unrelated things as they work. Future work should explore ways to detect when those different types of narration are occurring, and make use of their content as appropriate, for example highlighting moments where artists answer questions, or building a mapping of ways in which artists describe their software actions in natural language.

Chat logs may also be a useful data source to include in future work. Though the content of live chats tends to be very noisy, especially on popular channels \cite{Hamilton2014}, the frequency of chat posts at a given time can indicate exciting or interesting moments \cite{Pan2016}. Chat post frequency could therefore be used as another criteria for ranking clips. In addition, our formative work found that questions asked in the chat often go unanswered, usually because the streamer does not see them. Finding and highlighting such questions after the fact and encouraging the streamer (or other viewers) to answer them later could help keep viewers engaged once streams are archived (which several viewers in our formative surveys desired), as well as surface valuable feedback and ideas for the streamer.

\subsection{Other Ways to Generate Short Clips from Long Videos}
This work focused on generating tool-focused clips, inspired by prior work \cite{Grossman2010a, Lafreniere2014} and the natural mapping between tool use and user behaviour. However, while tool-focused clips are good for showing users how a tool works \cite{Grossman2010a}, they may not be the most inspirational types of clips one can generate from long videos. Other methods could include breaking videos into higher level sub-tasks (based on pauses in tool activity and narration content), or extracting clips where the artist describes a particular technique or answers a question. As one of the exciting parts of watching a live stream is seeing an artist go from a blank canvas to a finished work, another option could be to shorten the entire video down to a short summary. This could be done by removing sections where the artist takes pauses, talks with no actions, and switches applications; and by adapting existing techniques for creating short summaries from long videos (\textit{e.g.}, \cite{Truong2007}). 

\section{Conclusion}
This chapter explored a growing form of creative videos very different from traditional tutorials: live streams. We found that creative live streams are a good source of both learning and inspiration, and introduced an approach for bringing inspiration into the context of creative software users' workflows. 

RePlay, ReMap, and LiveClips all leveraged \textbf{visual media} in the form of screencast videos for providing contextual support towards understanding creative processes. But for users who just want to get a task done without worrying about the process behind it, videos can be too tedious and detailed. The next two chapters explore how two different types of resource -- \textbf{executable code} and \textbf{written text} -- can help people quickly and easily achieve creative outcomes.


\section{Acknowledgements}
We thank Tricia Ngoon, Kandarp Khandwala, and Nicolas La-polla for their help with live stream analysis, and our study participants for their insights. This work was supported in part by NSERC, Adobe Research, and NSF award \#1735234.

This chapter, in part, includes portions of material as it appears in \textit{Sharing the Studio: How Creative Livestreaming can Inspire, Educate, and Engage} by C. Ailie Fraser, Joy O. Kim, Alison Thornsberry, Scott Klemmer, and Mira Dontcheva in the Proceedings of the 2019 on Creativity and Cognition (C\&C '19). The dissertation author was the primary investigator and author of this paper.

This chapter, in part, includes portions of material coauthored with Andy Edmonds and Mira Dontcheva. The dissertation author was the primary investigator and author of this material.
