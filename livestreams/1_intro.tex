\section{Introduction} % or something more descriptive

%Browsing and exploring inspiring examples \cite{Shneiderman2007, Shneiderman2002, Greene2002, Herring2009, Bawden1986} and interacting with and getting feedback from other people \cite{} are both key parts of the creative process. Creative communities have existed for a long time to address these needs. Such communities often organize around sharing portfolios or completed projects (\textit{e.g.,} 500px, Behance and Dribbble\footnote{\url{500px.com}, \url{behance.net}, \url{dribbble.com}}). In recent years, creative communities have begun to form around a new medium for sharing not only final work, but the process behind it: livestreaming. 

%authenticity is the hot commodity now
%tech offers a window into a kind of authenticity that used to be harder to get
%\xxx{vignette sooner?}
Artists communicate and share their creative work through online \& on-land galleries, communities, and social media \cite{Kim2017}. Some also share works-in-progress, how-to tutorials, and videos describing the process that leads them to a final product. Prior work has shown that seeing and sharing the creative process is beneficial for creativity \cite{Kim2017}. However, these highly curated windows into process require time and effort for the creators to produce and share. 
%To create a good video tutorial, one has to set up an appropriate project, practice, record, and finally edit out mistakes or irrelevant parts \xxx{<-- xxx says move later but where?}. And works-in-progress communities are harder to find, as artists hesitate to mix works-in-progress with final work in their portfolios \cite{Kim2017}.\xxx{fine to move these two sentences to later. Not sure where. Can comment them out for now.}

\stepcounter{footnote}\footnotetext{Sources for video screenshots, from left to right: \href{http://bit.ly/2SK5zWE}{\nolinkurl{bit.ly/2SK5zWE}}, \href{http://bit.ly/2Bv9Y69}{\nolinkurl{bit.ly/2Bv9Y69}}, \href{http://bit.ly/2SJYFRa}{\nolinkurl{bit.ly/2SJYFRa}}, \href{http://bit.ly/2TK12Rq}{\nolinkurl{bit.ly/2TK12Rq}}}

Many artists have begun to broadcast live video as they work on graphic design, crafting, drawing, or music through platforms like Twitch and YouTube\footnote{\href{www.twitch.tv}{\nolinkurl{twitch.tv}}, \href{www.youtube.com}{\nolinkurl{youtube.com}}}. Livestreaming allows creators to share their unedited process \emph{while} they work. These videos usually feature the artist's workspace, %(\autoref{fig:livestreaming_view}a)
a view of their face,
%(\autoref{fig:livestreaming_view}b)
and their audio narration as they work (\autoref{fig:livestream_examples}). Viewers get an inside look into the creative process, learning from artists' decisions, mistakes, and surprises~\cite{Faas2018, Haimson2017}.  
%artists make along the way. 
Some also interact with artists directly via text chat. 
%Livestreaming shows everything, including mistakes, surprises, and dull moments . 
%Livestreaming has grown as a medium for sharing all kinds of content, from video gaming to travelling.
%Livestreaming has democratized the practice of sharing process, allowing anyone with a camera and internet to share their own process with the world.
%Livestreams provide a unique form of social engagement, allowing both one-to-many interaction of streamers to viewers, and many-to-many interaction among viewers \cite{Hilvert-Bruce2018, Hu2017}. 
Communities have formed around creative livestreaming, including dedicated platforms such as Picarto\footnote{\href{www.picarto.tv}{\nolinkurl{picarto.tv}}}.
Livestreaming democratizes the studio-apprentice model, enabling anyone to see experts' in-context choices by working alongside them \cite{Schon1985}. 
%Now anyone can sit in Leonardo da Vinci's studio and learn from the best. %something about making creative process accessible.

This paper seeks to understand what makes creative livestreaming so appealing for those who stream and those who watch.
%Some of these can include creative livestreams, \textit{e.g.,} programming is often considered creative, and many cultural heritage and knowledge-sharing activities are also creative, such as traditional calligraphy or drawing tutorials. But this still leaves a wide range of creative activities being livestreamed that are not yet understood. 
We provide the first broad look at the range of creative activities people stream. 
%By understanding creative livestreams, we can broaden our understanding of how to design streaming tools for a large and growing portion of the livestreaming community. 
Perhaps the three most popular genres for livestreams are video gaming \cite{Pellicone2017, Lessel2017, Sjoblom2017, Hamilton2014}, programming \cite{Faas2018, Haaranen2017}, and lifestyle  \cite{Lu2018a, Tang2016}. This paper looks at the use of livestreaming to share the process behind creating an original artifact. We use Twitch's definition of creative work: ``visual art, woodworking, costume creation, prop building, music composition, or any other process in which you entertain and connect around a creative activity'' \cite{Moorier2015}. These activities focus on creating a novel artifact, unlike typical video games or lifestyle streaming.

% YO:
%maybe also: previous work focuses more on building communities and performing in order to establish a certain audience. but not all creative streams do that, many just want to share their process
%freelancers are home a lot, many of their viewers are also freelancers or aspiring professionals, livestreams let you stay connected / have a coworking space to stay connnected with others while physically alone
%People go to Live streams to feel less alone at home working on their own thing. Since we know many people work while they are watching/listening to live streaming we can think of it as a virtual co-working space where you are working with your friends?

%The real-time interactivity and live participation that livestreams provide are engaging for viewers because they can be a part of the livestream \cite{Wohn2018}, and have the opportunity to change its outcome \cite{Lu2018a}.

%We find that streamer needs and viewer expectations differ considerably based on the type of creative stream, and creative livestreaming requires considerable preparation and strategic attention management.

We explore three main questions: 
\begin{enumerate}
    \item \textbf{What are creative livestreams?} For a general sketch of creative livestreams, we present a content analysis of a sample of livestreams that illustrates the range of content people stream and the different types of creative livestreams.
    \item \textbf{Why and how do people stream creative work?} Which parts of their process do they stream? To understand streamers' motivations, processes, and challenges, we present findings from interviews with 8 creative livestreamers and compare their experiences with streamers in other domains.
    \item \textbf{Why do people watch creative livestreams?} To understand the audience these streamers reach, we present findings from three online surveys with 165 viewers that highlight learning and inspiration as key motivators, with entertainment and community close behind.
\end{enumerate}

We found that viewers often seek to learn and be inspired from creative livestreams. Notably, inspiration is a much more prominent theme compared with prior work in other livestreaming domains such as gaming. However, many streaming platforms are not designed to support these goals. Audience engagement is particularly important for streamers, but can be difficult to achieve because of the conflicting goals different viewers may have and the deep focus creative work often requires.
% if authenticity is biggest takeaway, highlight it earlier. more problem solving in intro rather than taxonomizing.
% how do we expose authentic energy of people to mobilize the next gen of creatives?
% mistakes, liveness, risk, connection