\section{Open Questions and opportunities}
This paper's surveys and interviews uncovered the many goals and motivations streamers and viewers have for creative live\-streams. We also found that existing platforms do not support all these goals or offer help when goals conflict. We highlight areas for future research by asking three open questions.

\subsection{How might creative livestreams better engage viewers?}
In line with prior work, we found that creative streamers primarily interact with audiences through live text chat. Most interview participants mentioned difficulty keeping up with this chat, even though these streams are generally much smaller than video gaming streams. Sometimes, conflicting viewer goals can hinder the chat experience. Learners' questions can get lost in the many lines of text written by viewers who are there for social engagement. 
Streamers often enhance chat interaction using chat bots (one popular example is Nightbot\footnote{\href{https://nightbot.tv/}{\nolinkurl{nightbot.tv}}}) and install widgets to provide contests and other interactivity, but these take extra work to create, integrate, and manage.

\subsubsection{Augment chat functionality}
One approach for enhancing streamer-audience engagement might be to provide separate channels for different types of chat (as one \textit{S2} participant suggested). For example, learners could post questions in one channel while social banter happens in another.

Another approach could be to design more ways for viewers to communicate beyond text. Novel livestreaming interfaces allow audiences to participate in video games alongside a streamer, by drawing directly on the streamer's screen to suggest moves and voting on the streamer's next move \cite{Lessel2017}, or even participating directly in the game as a side character \cite{Glickman2018}. Creative livestreams may especially benefit from similar interactions. For example, viewers could annotate a streamer's work directly to ask a question about a particular section or provide feedback. Streaming platforms could also make it easier for streamers to set up polls without needing to spend too much extra time preparing them (\textit{e.g.,} detecting when the streamer poses a question and automatically creating a poll).

\subsubsection{Democratize the role of a host}
As our interviews demonstrated, having an extra person present on a stream as a host can be immensely helpful for artists. Having someone always watching the chat can alleviate this responsibility from the streamer when there are a large number of viewers, but even when viewership is small, having someone to ask questions and engage the artist in discussion can help keep a stream interesting. While moderators can address some of these challenges, by current conventions they typically do not, and not all streamers have the time or experience to find and train reliable helpers (\textit{e.g., P2}). How might we democratize the experience of having supportive hosts or facilitators?

One solution could be to take advantage of the auditory modality to mitigate the limited attention and screen space that streamers have when focusing on creative work. \textit{P4} suggested a text-to-voice service to read out chat messages so he doesn't have to look up from his work to answer questions, but noted that such a service would need to understand his own preferences so it could appropriately ``triage'' the chat, highlighting only important or relevant questions and comments. Such a tool might also help streamers feel more like they are participating in a conversation rather than one-way communication.

Another challenge is to create tools to help streamers troubleshoot their technical setup in lieu of trusted moderators or hosts. While streaming from mobile platforms has become as easy as pressing ``go live,'' many interview participants described spending a lot of time experimenting with technical settings to ensure that screencasts and camera views are clear and detailed, audio is on and good quality, and background music is at an appropriate volume. Three interview participants mentioned that a system that could automatically help with this setup (or give feedback on the quality of their setup) would save substantial time and effort. 

\subsubsection{Allow searching by goal}
Future work might explore how to match audiences to the right streams in the first place, like Sj{\"{o}}blom \emph{et al.} suggest for gaming \cite{Sjoblom2017a}. Current platforms typically allow viewers to find streams based on textual metadata, like the category and title. Platforms could allow streamers to make their goals for a stream explicit and searchable, so that those seeking to learn new skills could easily find Teaching streams, and those seeking to hang out with others could directly visit Socializing streams. In addition, platforms could enable or disable different modes of audience interaction depending on a stream's goals. Future research could explore what kinds of audience interaction best benefit different types of streams.
%(\emph{e.g.,} voice comments recorded by the audience may be suitable for Socializing but disruptive for Teaching).

%- looking for the right kinds of streams -- making goals explicit. Another strategy is to help people find the right streams in the first place.

% - professional vs. amateur goals
% - for amateurs, those doing art as main gig vs. side gig goals
% - say a bit about finance / money / how that changes things
% - viewers don't always get their questions answered
% - little ways for streamers to support audience involvement, platforms don't provide many ways for audiences to participate \cite{Glickman2018}
% - interaction is currently only through chat (what if you can't type? what if you miss the moment? what if you need to communicate visually?)
% - make sure to mention chatbots/games here?
% -- Current model of one linear chatroom makes peer learning hard - hard to extract higher level summaries and there is only one way to participate in the streamer's process --> \cite{Miller2017, Lu2018} \\

%\subsection{How might we support sharing different parts of the creative process?}
\subsection{How might we make creative work more ``performable''?}
Several interview participants mentioned that they were not comfortable streaming certain parts of their creative process, because they worried it would not engage the audience, or because it required their full focus. They would instead work on these parts offline to prepare for streams. Programming streamers face a similar tension between sharing their realistic process (including difficult and less-exciting parts like debugging) and keeping the audience entertained \cite{Faas2018}. Some artists (like \textit{P4}) are comfortable sharing their entire process from the beginning. For many viewers, it can be inspiring and educational to watch an artist go through the early ideation stages, but these parts of the process may need extra work to explain to audiences,
%not lend themselves as well to visual sharing, 
as they feature a lot of internal reflection and messy iteration \cite{Schon1983}. It is possible that more automated facilitation (as discussed previously) may allow artists to focus more on their work when it needs their full attention. How else might livestreaming platforms better support sharing \textit{all} parts of the creative process? Are there ways to make the early stages of creative work more ``performable'' for audiences? 

\subsection{How might we support watching livestream archives?}
%For the Making and Socializing types of livestreams, this seems to work reasonably well, as the main purpose of those streams is to interact with the audience in real time. However, for Teaching livestreams this can mean that valuable knowledge gets lost in the archives.

%Platforms currently tend to handle livestream archives uniformly, without considering the possible value a stream may have for future viewers.

On some platforms, such as Instagram, livestreams disappear shortly after they finish. On others, like YouTube and Facebook, they are re-playable archives that show up in search results alongside other videos. In between, platforms like Twitch archive videos for a limited time (14 or 60 days depending on account type); these archives are rarely re-watched  \cite{Jia2016}, as the affordances for finding them are limited.
%Indeed, Jia et al. \cite{Jia2016} found that most video game streams are forgotten soon after they are archived. 
While popular livestreams on YouTube continue to accrue views after they are archived, several survey participants mentioned the viewer experience is poor because the videos are long, have limited navigation, and include long periods of downtime and conversation with the then-live chat \cite{Lu2018}. Twitch viewers can create ``clips'' and streamers can create ``highlights'' of interesting moments, but they must remember to do so, and such moments can seem out-of-context when viewed on their own.

To make the most of the knowledge and experiences shared in creative livestreams, future work might explore how livestream archives could be more interactive and navigable. A relevant example is StreamWiki \cite{Lu2018}, where viewers collaboratively summarize a stream as it occurs. What might such a summary look like for a creative stream? How can we capture the moments of insight and inspiration for archive viewers without requiring them to watch all the downtime and unrelated conversation?
In addition, live summarization would be useful for both replays and viewers entering a stream in the middle. Automatic summary could help viewers catch up quickly and help them get a sense of whether a stream fits their goals.

%somewhere show that these are conneted to other social media too like people share posts from livestreams