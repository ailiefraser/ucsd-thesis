Many artists working with both digital and physical tools livestream their process as they work on a project. Despite this, research on livestreaming has primarily focused on gaming, pedagological, and software engineering use cases. As a result, livestreaming platforms are not always designed with creative work in mind. We present a first comprehensive look at creative livestreams, a growing medium where people share a window into their work as it unfolds. Through content analysis of livestream archives, online surveys with 165 viewers, and interviews with 8 streamers, we address current challenges in online creative livestream communities and compare them with prior observations of livestreaming in other domains. We find that, unlike other domains, one of the main reasons viewers watch creative livestreams is for inspiration and motivation. One dominant challenge that emerged for streamers was balancing between their work and interacting with viewers. This is especially difficult for creative work, which often requires the artist's full attention. We propose opportunities for future research to better support the needs of viewers and streamers of creative work.


\section{Extra stuff in case we need it}
\subsection{More on why seeing process is important and creative vs. procedural learning}
Websites like Behance and Dribbble tend to showcase finished projects, which give the viewers little to no insight about \textit{how} or \textit{why} a project was created. Recent work has shown that seeing the process behind an artist's work is beneficial for creativity, as it encourages self-reflection on one's own process and methods \cite{Kim2017}. Tutorial videos are one often-helpful medium for seeing the technical process behind how to do something. However, tutorials tend to show specific, often contrived, tasks, and impart mainly procedural knowledge \cite{Torrey2007}. Creative work involves not only \textit{procedural} skills but also \textit{creative} skills, such as making decisions about techniques, adjusting goals as a project develops, and trying multiple different versions of an idea. An increasingly popular way to share both the procedural \textit{and} the creative process is live-streaming. 

This paper presents a first look at creative live-streams, a burgeoning source of creative videos that impart creative knowledge, rather than only procedural knowledge. We argue that creative live-streams have unique benefits and challenges for both streamers and viewers. In particular, one challenge streamers face is balancing their attention between their creative work and interacting with the audience; as a result, some streams include hosts or moderators to help address and answer viewer questions. 



\subsection{More on related examples work}
Searching and browsing examples is an important part of the creative process \cite{Shneiderman2002, Shneiderman2007, Greene2002, Herring2009, Muller-Wienbergen2011, Bawden1986}. Lewis et al. \cite{Lewis2011} show that people can be primed to be more creative through exposure to examples before a creative task, and Kulkarni et al. \cite{Kulkarni} show that seeing examples early on in the process as well as interspersed throughout the process improves creativity.

Say a bit more about why these support creativity, and how videos of process are extra good

MixT: videos are better than static images or text for complex / motion visual stuff.

\subsection{What makes creative live-streams unique?}
We observed four major ways that live-streamed videos differ from other presentations of creative tasks.

First, live-streams provide unedited videos of people working on creative projects, showcasing not only their work, but also their mistakes, decisions, and iteration. There are often long chunks of time where the artist is silent, doing repetitive tasks, chatting about irrelevant topics, or away from the computer entirely. Live-streams can be more tedious to watch than tutorials, making quick searching and highlighting of key moments especially important. 

Second, this unedited view provides knowledge beyond just how to complete a specific task, whereas tutorial videos tend to provide clear, procedural instructions. Live-streams offer a similar visibility of work as in a design studio, enabling learners to see an expert's in-context choices \cite{Schon1985}. This is part of their main appeal in other domains as well, such as video-gaming \cite{Sjoblom2017}. Creative live-streams include advice on deciding \textit{which} tools or techniques to use, when to try one thing over another, tips for generating ideas, and general inspiration. This wisdom is especially valuable for creative tasks, offering a dimension beyond prior work on tutorials.

Third, live-streams are interactive. Viewers participate in the chat \cite{Pan2016, Hamilton2014} and the artist answers questions and contributes to the conversation. This means that viewers, even those not watching live, get to see how artists respond to questions, and these often lead artists to clarify what they've done in confusing or interesting moments.

Finally, making a live-stream does not require significant knowledge about video editing. Instead of planning out how to teach a particular task and piecing together a tutorial, presenters essentially do what they normally would, with the addition of a narrative explanation of what they are doing, similar to a think-aloud protocol. This opens up video sharing to expert artists who may not have previously felt comfortable or been willing to spend time making tutorials.
