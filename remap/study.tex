\section{ReMap Usage in the Lab and in the Wild}
To gain an initial understanding of how people use multimodal search for help, we conducted a think-aloud lab study with thirteen participants at a university. Participants were given a design to re-create in Canva and used ReMap to search for help as they worked.

Participants issued a total of 125 search queries; 118 used speech. Most participants used multimodal features to work and search or watch videos simultaneously.
%Many participants also mentioned that the markers on the video timelines made this much easier; they could easily skip between potentially relevant moments in the video rather than having to specify time intervals (which \cite{Chang2019} showed can be difficult). 
7 of 13 participants used deictic references at least once. Only 6 of 24 deictic references were successfully resolved to a name, mainly because some canvas objects were not recognizable by ReMap (\textit{e.g.,} graphs). More thorough accessibility labeling or integrated application plugins could improve this functionality.

Since ReMap's speech detection is always on, participants may encounter a ``Midas touch'' problem \cite{Jacob1990} of searching unintentionally. This happened occasionally but not frequently; future iterations of ReMap will more thoroughly explore the impact of such design decisions.

%3 queries were issued by mistake (not realizing they had spoken the ``search'' command), and 5 queries were issued before the participant had finished speaking. This is a challenge with a system that is always listening; on the other hand, needing to press a button to engage speech recognition can be burdensome.

We have also demoed ReMap at two open events, both in large spaces with over 100 attendees. ReMap's speech recognition requires a clear signal of the user's speech, mostly free of interference from sound output or background conversations. We have found commodity headsets to be sufficient, high-quality headsets to be optimal, and built-in laptop microphones insufficient. Overall, attendees who tried ReMap were excited about its multimodal features, particularly the potential of deictic resolution.