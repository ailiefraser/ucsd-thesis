\chapter{ReMap: Multimodal Help-Seeking}
\label{chapter:remap}
\begin{quote}
People are reluctant to search for help when they feel it will take more time than trial-and-error exploration, even with a contextual search system like RePlay. However, trial-and-error often takes a long time or does not result in a correct solution. Given this cognitive bias, how might we lower the barrier to searching for help? This chapter aims to make searching easier and faster through multimodal interaction. We introduce \textit{ReMap}, an extension to the RePlay system that allows users to speak search queries, adding application-specific terms deictically. Users can navigate ReMap's search results via speech or mouse. A lab study showed that ReMap helped people stay focused on their task while simultaneously searching for and using help resources. Users' experiences with ReMap also raised a number of important challenges with incorporating multimodal features into a help-seeking system.
\end{quote}

\section{Introduction: Making Search Feel More Natural}
Despite the ubiquity of online search, help-seeking remains a tedious and difficult process for many, because people are often reluctant to take their hands off the current task to search for help when they need it. Although RePlay's contextual search helped people spend less time once they decided to search, there was still an initial barrier that discouraged many from searching. Creating good search queries can be prohibitively difficult for novices, who may not know the domain vocabulary \cite{Russell2011}. It also switches the user's attention from the task at hand to the task of articulating their goal in a way that will match the resources they seek \cite{Tuovinen1999}. How might we lower the barrier to searching and make it easier for people to find the resources they need in the moment?

When people help each other, they use language, gestures, and shared context to communicate. In contrast, interacting with search engines to find help is still primarily text-based. This chapter explores how multimodal interaction might make searching easier, by enabling people to articulate their thoughts like they would with each other. Different types of information lend themselves better to different modalities; leveraging the strengths of multiple modalities and integrating them smoothly can be extremely effective \cite{Oviatt1999}. 

\begin{figure}[b!]
\centering
  \includegraphics[width=.7\textwidth]{remap/figures/interface.png}
  \caption[ReMap is a multimodal search interface for finding learning videos.]{ReMap is a multimodal search interface for finding learning videos. a) The user speaks their query. b) The user clicks on an image on the canvas while saying the word ``this.'' c) ReMap automatically changes the word ``this'' to ``image.'' d) ReMap highlights relevant moments by placing markers on the timeline of each video result.}~\label{fig:remap_interface}
\end{figure}

We introduce ReMap, a multimodal interface for users to search for learning videos using speech and pointing, without taking their hands (or mouse) off their current task. ReMap builds on RePlay (Chapter \ref{chapter:replay}), which uses relevant context from the user's activities to improve the relevance of search results. ReMap's multimodal help search demonstrates three main design insights:
\begin{enumerate}
\item To avoid context-switching, users can initiate and dictate a search using speech.
\item To avoid needing to learn or recall app-specific terminology, users can point at relevant software elements to include their names in the search query.
\item To simultaneously work on their task while following along with a tutorial, users can play, pause, and navigate video results using speech.
\end{enumerate}

A study with 13 participants found that ReMap allows people to stay focused on their task while help-seeking. The study, along with iterative prototyping and pilot testing, also raised a number of important challenges with incorporating multimodal features into a help-seeking system. We conclude this chapter by discussing ways future work can integrate multimodal search even more smoothly into creative tasks, further lowering the barrier to help-seeking.
\section{Related Work: Multimodal Interaction}
\subsection{The Power of Speech}
In 2016, Google reported that 20\% of all search queries on mobile Android devices were spoken \cite{Pichai2016}; that number has likely grown. Speech input is especially appealing on mobile devices, as people often use them while on the go and busy with other tasks \cite{Guy2016}; but even on desktop computers, voice assistants are often used for web search \cite{Mehrotra2016}. Speaking a search query is often easier and faster than typing it out, and it allows the user to ask a question in the way they might ask a friend; mobile voice queries tend to be closer to natural language and more often phrased as questions than text queries \cite{Guy2016}. Also, people tend to search by voice more when seeking audio or video results \cite{Guy2016}, supporting ReMap's approach of using speech to search for videos. Finally, speech may be more useful for users with specific goals: Laput \textit{et al.} \cite{Laput2013} found that when editing photos, speech was most useful when people knew exactly what they wanted to do. When people didn't know what they wanted, browsing a gallery of examples was more helpful as it allowed them to compare visual previews of potential effects before choosing one. Therefore, ReMap's main intended use case is for situations where users have targeted questions in the middle of a task, as RePlay was found to be most helpful in such cases and speech is likely to be especially beneficial. In other words, ReMap supports users seeking procedural help for achieving a specific outcome.

\subsection{Combining Modalities can Maximize Cognitive Abilities}
Combining input from multiple modalities (\textit{e.g.}, speech, gesture, touch) can reduce cognitive load for complex tasks \cite{Oviatt2015, Reeves2004}, make tedious tasks more efficient \cite{Oviatt1999, Salisbury1990, Karl1993, Hugunin1997}, reduce errors \cite{Oviatt1999}, increase precision \cite{Cohen1989}, and even make tasks more enjoyable \cite{Oviatt1999, Laput2013}. Such combinations work best when they maximize users' working memory by using different modalities for different types of information \cite{Oviatt2015, Kalyuga1999, Reeves2004, Stanney2004}. 
%Auditory output is helpful for giving the user information while they work \cite{Stanney2004, Grossman2007, Reeves2004}, and visual output and manual input are helpful for conveying spatial information \cite{Stanney2004, Reeves2004}. 
For example, telling the user about keyboard shortcuts for the operations they execute via auditory feedback increases the likelihood that users will retain them, as users' auditory working memory is otherwise unused when working in software \cite{Grossman2007}. Similarly, using speech to navigate tutorial videos while one's hands are busy with a physical task allows users to process both the video and the task simultaneously \cite{Chang2019}. ReMap similarly partitions different modalities between the user's creative task and ReMap's help-seeking interface (\autoref{fig:remap_modalities}), allowing users to speak search queries and video navigation commands.

\begin{figure}[t]
\centering
  \includegraphics[width=0.7\textwidth]{remap/figures/remap_modalities.png}
  \caption[ReMap partitions the above modalities between the user's creative task and their help-seeking task to maximize cognitive abilities. Users primarily use speech and listening to search for and navigate help videos while keeping their hands and eyes on their creative task.]{ReMap partitions the above modalities between the user's creative task and their help-seeking task to maximize cognitive abilities. Users primarily use speech and listening to search for and navigate help videos while keeping their hands and eyes on their creative task. The * indicates that the visual and motor modalities are not \textit{exclusively} used for the creative task; users can also transfer their visual and motor attention to the help resources when necessary, \textit{e.g.}, to watch a video or type a search query manually.}~\label{fig:remap_modalities}
\end{figure}

Pointing at objects and spatial locations is often easier and more natural than describing them in words only \cite{Reeves2004, Bolt1980, Larkin1987}. Similarly, using a screenshot of an interface element as a search query is easier than describing it in text \cite{Yeh2009}. When combined with speech, pointing allows people to communicate more precisely by using deictic terms (\textit{e.g.,} ``this'', ``here'') to refer to objects and spatial locations while pointing at them \cite{Bolt1980, Laput2013}. ReMap similarly allows users to point at interface elements and canvas objects in their creative software and reference them deictically while speaking a search query.

Multimodal input can be especially beneficial for creative tasks, where maintaining a flow state is important. Using different modalities for software-related actions allows users to keep their focus and hands on their creative work. For example, using speech to access tools instead of menus or keyboard shortcuts can help artists maintain creative flow and stay focused on their task \cite{Kim2019, Sedivy1999}. While doing digital painting, an artist could switch from the brush tool to the eraser by saying ``eraser'' instead of switching their attention and moving their hands to the toolbar or keyboard \cite{Kim2019}. Not surprisingly, much prior work exploring multimodal input has centered around creative tasks such as photo editing \cite{Laput2013}, graphic design \cite{Kim2019}, drawing \cite{Sedivy1999, Pausch1991}, data visualization \cite{Setlur2016}, and 3D modeling \cite{Sharma2011}. However, most of this prior work has used speech to execute commands, rather than issue search queries. This requires either defining a list of acceptable commands (which users must then memorize) or parsing natural language commands (which is prone to errors). This chapter combines insights from multimodal creative systems and voice search systems to explore how using voice to search might be useful in creative software.

\section{ReMap System Design and Implementation}
ReMap (\autoref{fig:remap_interface}) extends the RePlay contextual search system \cite{Fraser2019}. RePlay enables users to search for learning videos in context while working in software, and it highlights relevant moments in video results based on the user's context and query. While RePlay helps people find results faster, the attentional cost of switching to RePlay discouraged its being used as often as it could. ReMap lowers the switching cost and load by introducing three main improvements over RePlay: searching for help using speech, making deictic references in a search query, and navigating video results using speech commands.

\subsection{Searching for help using speech}
ReMap uses the Web Speech \textsc{api} to detect speech by opening a browser page in the background when launched. The page's JavaScript invokes continuous listening and sends the current phrase to ReMap's custom web server whenever a new word is detected.
Web Speech automatically determines when the user starts and finishes speaking, returning each phrase separately. If a phrase begins with \textit{``search''}, ReMap initiates a search (\autoref{fig:remap_interface}a), using the rest of the phrase as the query. Otherwise, it checks if the phrase matches any video navigation commands. If it does not, ReMap ignores it.

\subsection{Making deictic references in a search query}
Especially with new software, people are often unfamiliar with an application's vocabulary but can point at goal-relevant application elements. 
To alleviate the challenge of remembering app-specific terms, ReMap allows users to deictically reference interface elements and objects. If the user says \textit{``this''} or \textit{``that''} while clicking on a detectable element, ReMap replaces the pronoun with the reference element's name (\autoref{fig:remap_interface}b-c).

ReMap uses the MacOS Accessibility \textsc{api} to resolve element names. This \textsc{api} can get the name and description of any element with accessibility labels \cite{Fraser2019}. Many modern applications have labeled menus, buttons, and other interface elements. Some also label canvas elements (such as text boxes, images, and graphics) though many do not. The examples in this demo use Canva (\url{canva.com}) as the primary software. Canva labels most canvas elements and interface buttons.

While ReMap is detecting a speech query, it stores a list of every detectable element clicked. Once the user is finished speaking and ReMap has obtained the final query from the server, it replaces all occurrences of \textit{``this''} and \textit{``that''} with the element names in the order they were clicked.

\begin{figure}[b!]
\centering
  \includegraphics[width=\textwidth]{remap/figures/system.png}
  \caption{The ReMap system architecture. a) ReMap is a MacOS application that uses the Accessibility \textsc{api} to detect user context. b) ReMap connects to a web server, which opens c) a webpage for speech recognition, and d) a video player webpage for each result to embed in ReMap.}~\label{fig:remap_system}
  \vspace{-0.2in}
\end{figure}

\subsection{Navigating video results using speech commands}
The user can speak commands to navigate video results, inspired by Chang \textit{et al.}'s recommendations \cite{Chang2019}. ReMap currently supports the following commands: \textit{``play''} (plays the first or most-recently played video), \textit{``play \{next, previous, last\}''} (plays the next/previous/last video in the list), \textit{``play \{first, second, third, fourth, fifth\} video''}, \textit{``\{next, previous, repeat\} marker''} (skips to the next or previous timeline marker, or re-starts from the current marker), and \textit{``pause''} (pauses the currently playing video).


\subsection{Implementation}
ReMap is implemented as a MacOS Swift application (\autoref{fig:remap_system}a). It uses socket.io to communicate between the web server and the three client interfaces (the ReMap app, web speech engine, and video players). The web server (\autoref{fig:remap_system}b) is implemented in Node.js. The speech engine (\autoref{fig:remap_system}c) uses the Web Speech \textsc{api}, and the video player (\autoref{fig:remap_system}d) uses the YouTube Player \textsc{api} to load and control videos. When ReMap receives a video navigation command, it passes it to the server along with which video it applies to; the server then sends the command to the appropriate video player.

%ReMap generates a unique anonymous user ID for each computer it runs on. When it opens the speech engine and video player webpages, it includes this user ID as a parameter. For video player pages, it also includes that video's index in the result list. The server keeps track of these values for each client page that opens a socket connection, so that it can pass commands it receives to the appropriate client.
\section{Study: Using ReMap for a Graphic Design Task}
\label{sec:remap_study}
To gain an initial understanding of how people use multimodal search for help, we conducted a think-aloud lab study with thirteen participants. Participants re-created a given design in Canva (\href{https://canva.com}{\nolinkurl{canva.com}}) and used ReMap to search for help when necessary. Overall we found that despite some usability and implementation challenges, multimodal video search was helpful, allowing participants to stay focused on their task while simultaneously searching for help and navigating video resources. 

\subsection{Participants}
Thirteen participants were recruited from mailing lists and flyers at a university. 10/13 participants had at least some experience using voice assistants (\textit{e.g.}, Siri, Amazon echo) (mean = 2.3/5, 1 = never used, 5 = use every day). 6/15 participants had never used Canva before, and only one was very familiar with it (mean = 1.8/5, 1 = never used, 5 = very familiar).

\subsection{Procedure}
Participants were shown two images of an infographic design (\autoref{fig:remap_designs}) and were asked to choose one and re-create it as accurately as possible in Canva without using any of Canva's built-in templates. Participants were asked to use ReMap only (no web search) when they needed to search for help. Participants were given a brief tutorial on how to use ReMap, and were given three example commands to try to ensure they understood how it worked. They were encouraged to use speech as much as possible, but also had the option to type their search queries into ReMap's search field. Participants were asked to think out loud while they worked, and the experimenters captured audio and screen recordings. Participants were scheduled for 2-hour slots and were told to take as much time as they needed. Once they announced they were done (or if 1 hour and 45 minutes had passed), participants were asked a series of interview questions including prompts for Likert-scale responses to understand how they felt the task went, and whether they found ReMap helpful. Participants received a \$30 USD gift card for their time.

\begin{figure}[t!]
\centering
  \includegraphics[width=0.3\textwidth]{remap/figures/designA.png}
  \hspace{0.2in}
  \includegraphics[width=0.3\textwidth]{remap/figures/designB.png}
  \caption{Study participants chose one of the above two infographic designs to re-create in Canva, using ReMap to search for help.}~\label{fig:remap_designs}
\end{figure}

Both infographics were designed to require several operations that are not straightforward in Canva, to increase the likelihood that participants would have to search for help. These included: cropping an image in a circle, masking an image in text, creating a bar chart given a spreadsheet of data, making a swish shape behind text (\autoref{fig:remap_designs} left only) and adding a shadow to text (\autoref{fig:remap_designs} right only).

ReMap's speech recognition requires a clear signal of the user's speech, mostly free of interference from sound output (\textit{i.e.} from videos) or background conversations. We have found commodity headsets to be sufficient, high-quality headsets to be optimal, and built-in laptop microphones insufficient. For the study, participants wore a high-quality headset.

\subsection{Results}
\subsubsection{Search Behaviour}
Participants issued a total of 118 intentional search queries, 111 of which used speech (\autoref{table:remap_results}). An additional 3 queries were issued by mistake (not realizing they had spoken the ``search'' command) and an additional 7 were issued before the participant had finished speaking (because the Web Speech \textsc{api} detected a pause). One participant did not search at all (the same participant that rated their familiarity with Canva as 5/5); the rest issued between 4 and 18 queries each. 

\begin{table}[b!]
\centering
\caption{Summary of each participant’s usage of ReMap. ``\# speech commands'' includes all play, pause, and marker navigation commands. ``\# manual commands'' includes all plays, pauses, seeking along the timeline, and clicking on markers.}~\label{table:remap_results}
\resizebox{1\textwidth}{!}{
\begin{tabular}{lllllllll}
\textbf{} & \textbf{\begin{tabular}[c]{@{}l@{}}Total \#\\ intentional\\ searches\end{tabular}} & \textbf{\begin{tabular}[c]{@{}l@{}}\# accidental or\\ pre-emptive\\ searches\end{tabular}} & \textbf{\begin{tabular}[c]{@{}l@{}}\# intentional\\ speech\\ searches\end{tabular}} & \textbf{\begin{tabular}[c]{@{}l@{}}\# unique\\ videos\\ played\end{tabular}} & \textbf{\begin{tabular}[c]{@{}l@{}}\# speech\\ commands\end{tabular}} & \textbf{\begin{tabular}[c]{@{}l@{}}\# manual\\ commands\end{tabular}} & \textbf{\begin{tabular}[c]{@{}l@{}}\# deictic\\ references\end{tabular}} & \textbf{\begin{tabular}[c]{@{}l@{}}\# successful\\ deictic\\ resolutions\end{tabular}} \\ \hline
P1 & 17 & 1 & 17 & 13 & 41 & 3 & 5 & 1 \\
P2 & 5 & 0 & 5 & 6 & 2 & 93 & 1 & 0 \\
P3 & 13 & 3 & 13 & 7 & 20 & 23 & 10 & 1 \\
P4 & 4 & 0 & 4 & 5 & 14 & 19 & 0 & 0 \\
P5 & 0 & 0 & 0 & 0 & 0 & 0 & 0 & 0 \\
P6 & 8 & 1 & 8 & 5 & 17 & 2 & 0 & 0 \\
P7 & 7 & 1 & 7 & 6 & 24 & 10 & 0 & 0 \\
P8 & 3 & 1 & 3 & 2 & 4 & 1 & 0 & 0 \\
P9 & 8 & 1 & 6 & 6 & 0 & 49 & 0 & 0 \\
P10 & 16 & 0 & 13 & 5 & 10 & 5 & 3 & 0 \\
P11 & 10 & 1 & 10 & 11 & 9 & 51 & 1 & 1 \\
P12 & 11 & 0 & 11 & 4 & 8 & 16 & 1 & 1 \\
P13 & 16 & 1 & 14 & 7 & 18 & 0 & 3 & 2 \\
\textbf{Total} & \textbf{118} & \textbf{10} & \textbf{111} & \textbf{77} & \textbf{167} & \textbf{272} & \textbf{24} & \textbf{6}
\end{tabular}
}
\end{table}

61/118 queries (52\%) were new queries and 34/118 (29\%) were reformulations of previous queries (i.e., rephrasing a query to find better results). The rest were either attempts to fix a failed deictic resolution or fixing a transcription error. 55/118 queries began with either the phrase ``how to'' (42/118), or the phrase ``how do I'' (13/118). 

8/111 speech queries included a speech recognition error. One participant tried to rephrase their query after speaking it twice, leading to long queries with some repetition (\textit{e.g.}, ``resize this without aspect ratio without keeping aspect ratio''). 6 of the 7 non-speech queries were a manual revision of a previous speech query (either error fixes or reformulations). 

7 of 13 participants used deictic references at least once. Only 6 of 24 total deictic references were successfully resolved to a name, mainly because some canvas objects were not recognizable by ReMap (\textit{e.g.,} graphs). 23/24 deictic references referred to objects on the canvas (\textit{e.g.}, text boxes, images, and graphs); the other one referred to a button on the toolbar. This reinforces RePlay's study finding (\autoref{sec:replay_study}) that participants mostly made action-oriented queries, rather than queries about tools or commands, and highlights the need for canvas elements to be recognizable by ReMap.

\subsubsection{Video Navigation Behaviour}
Participants watched a total of 77 videos (\textasciitilde6 videos per participant). In total, participants played videos by clicking on them 60 times, and by saying one of the ``play'' commands 61 times. Participants paused manually 60 times, and via the ``pause'' speech command 46 times. Participants manually seeked to points in the video by dragging on the timeline 124 times, by clicking timeline markers 28 times, and by using the ``next/previous/repeat marker'' commands 60 times. In general, most participants seemed to exhibit a preference for either speech commands or manual navigation of videos (\autoref{table:remap_results}). This highlights the importance of providing both options in a multimodal interface, especially as peoples' preferences and needs may change depending on their task or environment \cite{Reeves2004, LaViolaJr.2014}.

\subsubsection{Participant Feedback}
As \autoref{fig:remap_boxplot} shows, participants generally found all four of ReMap's multimodal features moderately helpful. Based on our observations, most participants used the multimodal features to work and search or watch videos simultaneously. They confirmed this during the post-task interviews, with several participants explicitly mentioning multitasking as a benefit of ReMap's multimodal features. For example, one participant described how ReMap helped them be more efficient than their current strategies for help-seeking:

\begin{quote}
\textit{``When I'm designing graphics myself sometimes I get stuck, so I will have to stop everything and go to YouTube or Google to find it, but if I'm able to work while listening or ... watching it on the side, through voice, I think it will help with efficiency ... `cause at one point I was able to ... move the things around while listening for the things I need.''} --- P4
\end{quote}

\begin{figure}[t!]
\centering
  \includegraphics[width=0.7\textwidth]{remap/figures/boxplot.png}
  \caption{Distribution of participants' ratings of ReMap’s multimodal features. 1 = not helpful at all, 5 = very helpful. Bold lines represent median values, and boxes illustrate interquartile ranges. }~\label{fig:remap_boxplot}
\end{figure}

\textbf{Timeline markers:}
Navigating the timeline markers was rated as the most helpful multimodal feature on average, and was the most common answer participants gave when asked what their favorite feature of ReMap was. This is mainly because it supported participants in multitasking:

\begin{quote}
\textit{``I ... found it helpful because while it was switching I could multitask, and I could just tell it `hey, go to the next marker', and then I would try to do stuff on my own on the side until it plays.''} --- P6
\end{quote}

\begin{quote}
\textit{``If I couldn't use speech then i would have to ... actually click on the videos, but since I'm already working on this part of the screen, then [speech is] just more convenient.''} --- P7
\end{quote}

The markers themselves were useful (as the RePlay study (\autoref{sec:replay_study}) also showed) as they provided a shortcut to potentially relevant moments in the video. The speech commands for skipping between them made it easy for participants to back-up or fast-forward the video to a reasonable point. This is easier than having to specify a specific time interval to skip between, which Chang et al. \cite{Chang2019} showed can be difficult for people following along with video tutorials. However, some participants did mention that they preferred using the mouse to hover over markers before selecting them, so they could see the caption previews (\autoref{fig:replay-green_markers}): \textit{``I don't really know what they are until I scroll over them''} (P11).

\textbf{Play/pause:}
Some participants also found speech for playing and pausing videos to be helpful, as it allowed them to follow along with videos at their own pace, which prior work has also demonstrated the importance of \cite{Chang2019, Pongnumkul2011}.

\begin{quote}
\textit{``It's really helpful because if I hear the info I want and I'm kind of following along I can pause it, take the instructions, apply it, and then ... continue watching ... without having to stop the fluidity of work.''} --- P8
\end{quote}
\begin{quote}
\textit{``It was nice to be able to keep track of the video in the background while also starting to think through the next thing and having my focus on the other window.''} --- P13
\end{quote}

Other participants preferred using the mouse to play and pause the videos as they felt it was just as fast or easy, and didn't require them to remember the correct wording of the commands.

\textbf{Speaking query:}
Many participants found it helpful to speak their query out loud rather than type it, as it also supported multitasking and allowed them to stay focused on the task in Canva. Some also said that it was easier or faster than typing. However, some other participants found it more difficult to speak their query, as they didn't always know exactly what to say when they started, and they were not used to speaking out loud to search. As P13 described, \textit{``I had a lot of trouble getting my queries totally straight or thought out in my head before starting to speak them.''} One common difficulty encountered with ReMap's implementation was that it would cut off a participant's query and issue the search before they were done speaking, often because they paused slightly while thinking of what to say. This happened 7 times and was due to the Web Speech \textsc{api} prematurely detecting the end of a phrase. As one participant described, it made them feel rushed to finish speaking:

\begin{quote}
\textit{``It kept rushing me to think about what I wanted to search for ... Usually when I search by typing, I type out half of what I want and then think about ... what to type for the rest, whereas here as soon as i said `how to add' I had to immediately know what I was searching.''} --- P6
\end{quote}

\textbf{Deixis:}
The 7 participants who attempted using deixis in their queries mostly found it helpful, although many noted that it needed some improvement to be useful. It helped participants include words they didn't know in their queries and was often easier and faster than saying or typing the words explicitly. However, it is likely most helpful in cases where users do not know the terminology, and Canva's interface has a relatively straightforward vocabulary. For example, P1, who rated their prior experience with Canva as 1.75/5, said \textit{``it might be more helpful for things that you don't know the exact terminology for, but I think I knew some of the terminology so just saying it felt faster.''}

ReMap's deictic resolution also suffered some implementation and usability challenges: deictic references were only successfully resolved 25\% of the time, usually due to objects on the canvas not being recognized by the Accessibility \textsc{api}. Two participants also pointed out that their eyes naturally went to the search field where their query was appearing while they spoke it, which made it difficult to look at Canva instead to reference elements: \textit{``even though i was trying to click here i was focusing on [the search field]''} (P9).

\textbf{Overall:}
Finally, when asked if participants could see themselves using ReMap in their daily lives, 9 participants said yes (3 saying for certain tasks only, and 2 saying if the technology improved first). The 4 participants who said no shared reservations about using speech in public or around people, and some said they simply prefer typing.

\section{Discussion: The Potential of Multimodal Help Search}
Overall, most participants were positive about multimodal help search, and many of the challenges they exhibited stemmed from either implementation issues or unfamiliarity with using speech and pointing for search and navigation. It is likely that an improved implementation combined with more time spent using ReMap would further increase users' success. This section outlines how ReMap's usability and implementation could be improved based on the study findings.

\subsection{Usability Challenges With Speech for Search}
\subsubsection{The Midas Touch Problem: How best to initiate a search?}
Since ReMap's speech detection is always on, participants occasionally encountered a ``Midas touch'' problem \cite{Jacob1990} of searching unintentionally. This is a challenge with a system that is always listening; personal assistants that can be activated via a voice command (\textit{e.g.}, ``Hey Siri'') cause similar accidents \cite{Hern2019}. On the other hand, needing to press a button to engage speech recognition can be burdensome, and in our study, accidental searches only happened 3 times out of all 121 speech queries issued. ReMap's current design (like many personal voice assistants today) was informed largely by the goal to make searching as quick and easy as possible, but for something as intentional as a search query, the added burden of pressing a button may still be more convenient than the risk of error. Future work should compare ReMap's current design of speaking a keyword to search with a button or keyboard shortcut to activate speech detection. 

\subsubsection{Pre-emptive searching: How to indicate a query is finished?}
Several participants were frustrated by ReMap issuing a search before they had finished speaking their query, usually because they had paused briefly to think about what to say next and the Web Speech \textsc{api} interpreted this as the end of a phrase. Most voice assistants have this same functionality of automatically detecting when the user is finished speaking (\textit{e.g.}, Siri, Alexa), but they do sometimes fail and cut the user off early. A study by Jiang et al. \cite{Jiang2013} found that these ``system interruptions'' accounted for about 10\% of all transcription errors. In our study, they accounted for about 24\% (7/29) of all transcription errors, including speech recognition errors and unintentional searches. Similar to the alternatives proposed above, future work should explore whether it is preferable to require the user to explicitly press a button, say a keyword (\textit{e.g.}, ``go''), or use a keyboard shortcut to indicate that they are finished speaking. Again, this adds an extra step to the process but would likely prevent errors and remove the pressure some participants felt to rush to finish speaking.

\subsubsection{How much to show users and when?}
A few participants noted that while they were speaking their query, they would watch its transcription appear in ReMap's search field. This may have been distracting, as it took their attention away from the task at hand and made deixis feel less natural. Indeed, Kalyuga \textit{et al.} \cite{Kalyuga1999} showed that splitting one's attention in the same modality increases cognitive load; in this case, user's visual attention was split between the software and ReMap's search field. We have since updated ReMap with a setting that can be toggled to specify whether to show the query as it is transcribed or only when it is finished; future work should compare these to see whether one consistently performs better, or if personal preference should dictate the setting. Seeing one's speech transcribed in real time (as many mobile voice assistants do) assures the user that the system is actively listening and gives them immediate feedback as to whether their speech has been accurately transcribed. But in the case where the search interface is not the primary application being used, it may add more distraction than the assurance is worth. 

\subsubsection{How to correct mistakes in a speech query?}
In general, the problem of making a correction to a speech command or query is a difficult one \cite{Myers2018, Jiang2013, Paek2008}. Errors may be caused by the system (\textit{e.g.}, speech transcription errors) or by the user (\textit{e.g.}, saying a word they didn't intend). When such errors occurred with ReMap, participants mostly resorted to either editing the query manually by typing or repeating the entire query over again, echoing previous findings \cite{Myers2018, Jiang2013}. One participant corrected their query in real time by repeating it while speaking, much as one might correct oneself in normal conversation, which led to the entire utterance being transcribed as one long query. A smarter system might recognize this repetition and automatically use the corrected version only. Future work should explore how this and other methods for correction might be implemented. For example, Jiang \textit{et al.} \cite{Jiang2013} recommend letting users specify and repeat only a certain portion of a query that they want to correct. Shokouhi \textit{et al.} \cite{Shokouhi2014} have shown that (at least on mobile) people prefer not to switch between speech and text when correcting a query, so one option could be to support speech commands such as ``change X to Y'' that allow users to replace incorrect words using voice alone.



\subsection{Improving the Robustness of System-wide Contextual Support}
ReMap contributes the first system-wide means for multimodal contextual help search. It achieves this by leveraging accessibility \textsc{api}s to obtain system-wide information about the user's actions. However, this approach only works when accessibility \textsc{api}s provide the necessary information, which they sometimes do not. This section discusses how future work might address these gaps.
% Our experience building ReMap led to several important implementation challenges that would need to be addressed before a system like this could be widely released.

\subsubsection{Limitations of current accessibility APIs}
ReMap is only able to detect interface elements that have been explicitly labeled with accessibility information by the application developer, which as both RePlay and prior work \cite{Hurst2010, Chang2011} have shown, varies widely across applications. This chapter's study also found that even in one of the most thoroughly labeled creative software applications (Canva), some accessibility labels were still missing (\textit{e.g.,} charts on the canvas). In general, accessibility labeling tends to be more common for interface ``widgets'' such as menu items and tools, and less common for the ``insides'' of applications such as editing areas or canvases \cite{Hurst2010}. However, there are existing efforts in the research community to make applications and systems more accessible using crowdsourcing and computer vision \cite{Guo2016, Guo2019StateLens}, so accessibility labeling may improve in the future.

ReMap's reliance on accessibility \textsc{api}s also complicates its setup and implementation substantially. As discussed in the Implementation (\autoref{sec:remap_implementation}), ReMap needed to be built as a MacOS application in order to access the MacOS Accessibility \textsc{api}, but this required building separate webpages to handle video playing and speech detection, as well as a custom server to communicate between the webpages and ReMap. In our pilot tests and study, this setup generally worked well, although occasional delays in speech transcription did happen. However, opening a page in the web browser to detect speech would be impractical if a system like ReMap were to be deployed more widely. How else might we enable both multimodal support and detection of contextual information?

\subsubsection{Leveraging Accessibility on Mobile Systems}
ReMap may work more reliably as a mobile application, because mobile operating systems such as Android and iOS tend to have more accessibility features built in and more standardized interface components in their applications. Prior work has shown how mobile applications can leverage accessibility information to enable programming by demonstration \cite{Li2017}; a mobile version of ReMap could similarly use accessibility to provide contextual support across tablet and smartphone applications (which are increasingly used for creative work, as evidenced by the increasing prevalence of mobile apps from companies like Adobe\footnote{\url{https://www.adobe.com/ca/creativecloud/catalog/mobile.html}}). In addition, speech and pointing are already common modes of interaction with mobile devices, which may make ReMap's multimodal interaction feel more natural. 

\subsubsection{Computer Vision for Detecting Interface Elements}
Prior work has used computer vision as an alternative to accessibility \textsc{api}s for recognizing interface elements \cite{Chang2011, Hurst2010, Dixon2010}. Future work should explore how computer vision might enable system-wide contextual assistance instead of relying on accessibility labels. Computer vision may in many cases be \textit{more} effective than accessibility as it could allow for higher-level interpretation of the elements being clicked. As our studies of RePlay and ReMap found, participants rarely found tool-level context useful for including in their search query. Even when accessing commands directly, Bota \textit{et al.} \cite{Bota2018} found that people often search for actions or outcomes rather than command names. ReMap's addition of canvas elements helped make deictic resolution useful, as most deictic references participants made were for canvas elements. However, even when such elements have accessibility labels, they tend to be very general. Higher-level semantics about the element may be more useful; for example, a user might reference a photo of a sky when searching for help, in which case knowing that it shows a sky might be more useful than just knowing that it is a photo. 

Of course, computer vision may also have limitations compared to accessibility \textsc{api}s; the description or meaning of something is not always apparent from its visual attributes alone. For example, many tools in creative applications are represented by icons only; their names are not displayed. Or, the action a tool performs may not always be apparent from its visible name.

\subsubsection{System-wide Support vs. Detailed Contextual Knowledge}
ReMap illustrates both the benefits and drawbacks of system-wide, application-general assistance. While it provides users with a consistent interface for finding help across multiple applications, it is not able to tailor itself to a specific application the way an integrated plugin could. We believe the benefits outweigh the costs, especially given the potential improvements mentioned above that could be made to ReMap's approach for gathering contextual information. Nonetheless, one potential middle ground could be to build ReMap as a browser extension (similar to CheatSheet \cite{Vermette2015}). As web applications are becoming more prevalent than desktop software (with some laptops running \textit{only} web applications, \textit{e.g.,} Chromebooks), this might be a fair compromise. A browser extension could use voice recognition directly via the Web Speech \textsc{api}, enable deictic resolution and context recognition by accessing a webpage's document structure, and provide consistent support across multiple applications (as long as they are on the web).


% maybe include below in paper...
% \subsection{Translating Multimodal Input Into an Effective Search Query}
% Since ReMap leverages a commercial search engine, it must ultimately produce a textual search query to send to the engine. This means translating the user's gestures, deictic references, and context into a single string, where every keyword has a roughly equal weight (depending on the search engine's particular algorithm). But what if a search engine could consider these different input sources individually? Might that improve the quality of results? For example, DeepStyle \cite{Tautkute2019} allows users to search for fashion items by providing both text input and visual examples; its custom search engine is able to compare visual and stylistic similarity of the user's input examples with its corpus. ReMap could similarly compare visual attributes of the user's interface and documents with online videos.
\section{Conclusion}
ReMap demonstrates multimodal interaction for quick, in-context help-seeking by leveraging the strengths of multiple modalities. Users can search for videos using speech, use deixis to include app-specific terminology, and use speech to navigate videos. Initial usage showed that ReMap helps people stay focused on their task while navigating help resources, but further research is needed to provide robust deictic resolution and fully explore its potential for help-seeking.
