\section{Conclusion}
ReMap introduces multimodal interaction for quick, in-context help-seeking by leveraging the strengths of multiple modalities. Users can search for videos using speech, use deixis to include application-specific terminology, and use speech to navigate videos. An initial study showed that ReMap helps people stay focused on their task while navigating help resources, and highlighted several important challenges with multimodal search. 
%Future work should explore how to make deictic resolution more robust and further lower the barrier to searching in context.

RePlay and ReMap demonstrated how bringing learning videos into the user's context can help people find procedural help while working toward a specific desired outcome. The next chapters in this dissertation break apart the goals of \textit{process} and \textit{outcome}, exploring how contextual systems can support users who desire only one or the other.

\section{Acknowledgements}
This chapter, in part, is currently being prepared for submission for publication of the material. C. Ailie Fraser, Julia M. Markel, N. James Basa, Mira Dontcheva, and Scott Klemmer. The dissertation author was the primary investigator and author of this material.
