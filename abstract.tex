Translating goals into concrete actions is a major challenge for people doing creative activities. Expert examples, tutorials, and videos abound online, helping people find inspiration and learn from the best. However, online resources are detached from the user’s work. The user must determine which resource is most relevant to their particular situation, and adapt it to their context. The sensemaking challenge of using help resources combined with the open-endedness of creative work can be tedious at best, and paralyzing at worst, preventing people from reaching their full creative potential.

My dissertation introduces methods for \textbf{curating existing expert help and demonstrations} and \textbf{presenting them to users in context}, with the goal of better supporting the iterative creative process and thus enabling people to produce better work. Leveraging existing expert-made resources enables people to learn from others more easily, and placing them in context promotes serendipity and unexpected connections.

When performing creative activities, people can be driven by processes and/or outcomes: sometimes they are interested in understanding the process behind creative tasks, while other times they wish to reach a particular outcome the fastest and easiest way possible. This dissertation presents four interactive systems that explore how three different types of expert resource -- \textbf{visual media}, \textbf{executable code}, and \textbf{written text} -- can help people accomplish one or both of the above goals at a level beyond what they could have achieved otherwise.

Specifically, \textit{RePlay} and \textit{ReMap} support both process and outcome by enabling in-task search and navigation of help videos. \textit{LiveClips} supports discovery and exploration of expert processes by embedding inspirational clips from live streamed videos into the user’s workflow. \textit{DiscoverySpace} helps novices discover and achieve expert outcomes by recommending action macros in-situ. Finally, \textit{CritiqueKit} combines adaptive suggestions with interactive guidance to help novices improve their outcomes in the moment.

Together, these systems and their evaluations demonstrate how curating and presenting expert help and demonstrations in-situ can help novices build confidence, accomplish tasks, and produce higher-quality creative work. My dissertation enables more people to reach their creative potential by lowering the barriers to getting started and completing projects.
