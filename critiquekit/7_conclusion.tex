\section{Conclusion}
Looking across the deployments and experiments, adaptive suggestions and interactive guidance significantly improved feedback while static suggestions did not offer significant improvements. These techniques were embodied in the CritiqueKit system, used by 95 feedback providers and 336 recipients. Although CritiqueKit's main goal was to help reviewers produce better outcomes quickly and in the moment, we suspect that the combination of interactive guidance and illustrative examples may also help reviewers learn and retain procedural about writing good feedback. Future work should evaluate this hypothesis and further investigate how best to create, curate, and display helpful suggestions. 

Much knowledge work features both underlying principles and context-specific knowledge of when and how to apply these principles. Potentially applicable feedback and review areas include domains as disparate as hiring and employee reviews, code reviews, product reviews, and reviews of academic papers, screenplays, business plans, and any other domain that blends context-specific creative choices with common genre structures. More generally, this work, together with Discovery\-Space, showed how contextual suggestions and guidance can help improve creative outcomes in the moment by leveraging and reusing existing expert work. We hope that creativity support systems of all stripes will find value in the ideas and results presented here.