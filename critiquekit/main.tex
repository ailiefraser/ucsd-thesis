\chapter{CritiqueKit: Interactive Guidance Techniques for Improving Creative Feedback}
Good feedback is critical to creativity and learning, yet rare. Many people do not know how to actually provide effective feedback. There is increasing demand for quality feedback – and thus feedback givers – in learning and professional settings. This paper contributes empirical evidence that two interactive techniques – reusable suggestions and adaptive guidance – can improve feedback on creative work. We present these techniques embodied in the CritiqueKit system to help reviewers give specific, actionable, and justified feedback. Two real-world deployment studies and two controlled experiments with CritiqueKit found that adaptively-presented suggestions improve the quality of feedback from novice reviewers. Reviewers also reported that suggestions and guidance helped them describe their thoughts and reminded them to provide effective feedback.

\section{Introduction: Feedback's Hidden Potential}
Feedback is one of the most powerful influences on learning and achievement [14]. Both giving and receiving formative feedback encourage self-reflection and critical thinking on one's work [24,31], especially in creative and open-ended domains such as design and writing [14,35]. The growing scale of many educational and professional settings increases both the importance and difficulty of providing sufficiently descriptive and personalized feedback. Good feedback can be hard to generate, and people are not consistently skilled in doing so [22,46]. Feedback is often too short, vague, and not actionable [20,40,45]. Even experienced reviewers don't always recognize when they are providing poor feedback or why it is ineffective [40].

This paper contributes two interactive techniques that improve feedback, their embodiment in the CritiqueKit system, and their evaluation through two deployments and two experiments.

\textbf{Interactive guidance of feedback characteristics.} CritiqueKit features a guidance panel with checkboxes that update as the reviewer gives feedback. A text classifier categorizes feedback as Specific, Actionable, and/or Justified as the reviewer types, providing them with an ambient awareness of their feedback quality and guiding them to improve their feedback. 

\textbf{Suggesting prior feedback for reuse.} CritiqueKit enables reviewers to reuse expert feedback, reducing experts' labor by scaling their feedback to similar work. These suggestions update and adapt based on the feedback's categorization to give reviewers targeted ideas for how to improve their comment and provide inspiration. 

Two deployment studies and two controlled experiments investigated the efficacy of these interactive techniques on the quality and characteristics of feedback. The first deployment examined how experienced reviewers (teaching assistants) reuse feedback in an undergraduate course. The second deployment examined how undergraduate students reuse feedback. The first experiment examined the impact of statically presented suggestions and interactive guidance on novice feedback. Finally, the second experiment examined the efficacy of adaptively updating suggestions in tandem with interactive guidance on novice feedback. We found that adaptively-presented suggestions improved feedback quality (\autoref{fig:critiquekit_exp2}). Reviewers found suggestions useful for inspiration, and the interactive guidance reminded them to ensure their comments met the criteria for effective feedback. This work provides evidence that interactive techniques such as suggestions and guidance can effectively scaffold the feedback process (See \autoref{table:critiquekit_all_results} for details).

\begin{figure}
\centering
  \includegraphics[width=0.6\textwidth]{critiquekit/figures/table1.png}
  \caption{Two deployments (DEP) and two between-subjects experiments (EXP) examined the efficacy of feedback reuse and interactive guidance. We found that interactive suggestions and guidance were most helpful for improving feedback.}~\label{table:critiquekit_all_results}
\end{figure}

\section{Related Work}
\subsection{Good Feedback is Actionable, yet Rare}
Rapid iteration is critical to the success of creative projects, from essays, to visual design, to buildings [5,35]. Receiving feedback early on is important for learners to test alternatives and course-correct [5,41]. Effective feedback is especially important in educational settings where novices are learning new skills and developing expertise. However, giving effective feedback is rarely taught [30]. As physical and digital classrooms increase in size, the demand for feedback outgrows the ability to adopt the ideal learning model of one-to-one feedback [2]. Instead, a one-to-many approach is utilized, where an expert provides feedback for multiple learners. Although learners most value expert feedback [9,27], the one-to-many approach is highly demanding on experts, and specific, actionable feedback for individuals becomes increasingly rare. 

In general, effective feedback is specific, actionable, and justified. Specific feedback is direct and related to a particular part of the work rather than vaguely referent [19,35,46]. Specific positive feedback also highlights strengths of the work and provides encouragement, so the recipient can tell they are on a good path [18,43,46]. Actionable feedback is important because it offers the learner a concrete step forward [35,40,43,46]. Simply pointing out a problem is not sufficient to help one improve [32,35,40,41]. Actionable feedback is often most helpful early in a project [4,43,46] because it may help people self-reflect and self-evaluate their work [8], prompting more revisions for improvement [6,42]. Lastly, justification is an important characteristic of feedback [19,28,46], but is arguably one of the hardest to understand or recognize [9]. Justified feedback contains an explanation or reason for a suggested change, which helps the learner understand why the feedback was given.

\subsection{Rubrics \& Examples Usefully Focus Feedback}
Rubrics [1,46] and comparative examples [19] are effective in structuring feedback because they beneficially encourage attention to deep and diverse criteria. Novices otherwise tend to focus on the first thing they notice, often surface-level details [12,17,20,46]. Viewing examples of past designs can lead to greater creativity and insights [21,26]; thus, showing examples of good feedback may spark ideas reviewers would not have otherwise considered [12,22,25]. Also, adaptive examples curated to match design features are more helpful than random examples in improving creative work [23]. 

Rubrics and other scaffolds require significant upfront manual work by experts who must carefully design a comprehensive rubric, curate a thorough set of examples, or decide how else to structure the feedback process. This paper investigates leveraging existing feedback to dynamically create rubric criteria. We hypothesize that showing reviewers previously-provided feedback can guide their attention to important aspects of the design. 
\subsection{Is Feedback too Context-specific for Practical Reuse?}
Schön persuasively argues that effective feedback should be context-specific and expert-generated [36]. He offers a vignette from architecture where the teacher suggests an alternative building to the student as an example of situated wisdom and its transfer. If Schön is right that this exchange requires both wisdom and context, does that mean that feedback reuse is infeasible? Within a given setting, project, or genre, common issues recur. Hewing to the principle of recognition over recall, we hypothesize that suggestions and guidance can increase novices' participation in context-specific exchanges. 

\subsection{Prior Systems \& Approaches for Scaling Feedback}
Existing approaches for scaling personalized feedback include clustering by similarity (\textit{e.g.}, for writing [3] and programming [10,15]). Gradescope [39] and Turnitin [47] allow graders to create reusable rubric items and comments to address common issues and apply them across multiple assignments. Gradescope binds rubric items to scores, which emphasizes grades rather than improvement. 

Other methods include automating the reuse of the solutions of previous learners. These methods work best when correct and incorrect solutions are clearly distinct, such as in programming [11,13] and logical deductions [7]. Automated methods have also found success with the formal aspects of more open-ended domains such as writing [3,34].  However, assessing the quality and effectiveness of creative work – the strength of a design, the power of a poem – is intrinsically abstract and subjective and lies beyond current automated analysis techniques. Also, little automated analysis exists for media other than text. For domains like design, human-in-the-loop analysis will remain important for quite some time. 

\subsection{Automatically Detecting Feedback Characteristics}
Although feedback is often specific and contextual [36], general characteristics can be automatically detected and used to help reviewers improve their feedback. For example, PeerStudio detects when comments can be improved based on the length of the comment and the number of relevant words [20]. Data mining and natural-language processing techniques can also automatically detect whether a comment is actionable or not, and prompt the reviewer to include a solution [29,45]. Krause et al. use a natural-language processing model to detect linguistic characteristics of feedback and suggest examples to reviewers to help them improve their comment [19]. These methods require a reviewer to first submit their comment so it can be analyzed, and then improve their comment after submission.

\section{RePlay System}
This section describes RePlay's user interface, context detection, and system architecture.
%RePlay is a desktop application that allows users to search for contextually-relevant video clips for any accessibility-enabled desktop application. It leverages large online video corpora, searching video caption text for relevant moments. The RePlay prototype investigates the efficacy and challenges of cross-app contextual search.

\begin{figure}[b!]
\centering
\vspace{-0.2in}
  \includegraphics[width=\textwidth]{replay/figures/green_markers.png}
  \caption{RePlay overlays green markers on the video timeline to indicate moments where the captions match a user's search term. Mousing over a marker shows a pop-up with an excerpt from the captions at that moment; clicking it plays the video from that moment. }~\label{fig:replay-green_markers}
%   \Description[A screenshot showing green markers overlaid on a video result in RePlay.]{A screenshot showing green markers overlaid on a video result in RePlay. The user has searched for ``add images to grid'' in Adobe XD, and the top video result is titled ``Learn how to quickly add images in Adobe XD''. The video timeline has green dots on it, and the mouse is hovering over one. A pop-up shows a caption excerpt that says: ``...spacing all the images are going to line...'' with the word ``images'' bolded.}
  \vspace{-0.2in}
\end{figure}

\subsection{User interface}
The RePlay panel (\autoref{fig:replay-interface}) can be positioned and sized as desired; its default size is 465px \texttimes 1055px. It is designed to fit next to the user's primary applications to minimize switching windows. The narrow pane makes videos small but easier to browse and watch in context. The interface comprises a status area, search field, and video results.

The status area updates as the user works, displaying the name of the last tool clicked and the current application (\autoref{fig:replay-status}). ``Tools'' refer to interface elements or commands within an application. The status area provides awareness of what context RePlay will use for search (\textit{i.e.,} so the user does not need to include the application name in their search query). When the user initiates a search, the status area updates to show the query (\autoref{fig:replay-interface}b).

As the user works, the search field updates with the name of the last tool clicked (\autoref{fig:replay-status}). Users can edit or delete it to form their own query. Pressing the \textsc{go} button or return key triggers a search. RePlay displays the top five resulting videos, each cued to start at a relevant moment. A two-line excerpt from that moment's caption appears below the video with query words highlighted in bold \cite{Hearst2009}.

Often, videos have multiple moments that may be relevant. RePlay renders green markers on the video timeline to indicate these moments. Mousing over a marker invokes a pop-up text area displaying a caption excerpt with words from the query in bold (\autoref{fig:replay-green_markers}). This pop-up obscures YouTube's default thumbnail pop-up but provides more useful information, as software videos tend to show an entire screen and shrinking this to a thumbnail makes it hard to see. Clicking a marker starts the video from that moment. 

\begin{figure}[b!]
\centering
\vspace{-0.2in}
  \includegraphics[width=\textwidth]{replay/figures/replay_status.png}
  \caption{RePlay's status area displays tool names after they are clicked and adds them to the search field. }~\label{fig:replay-status}
%   \Description[A screenshot of RePlay's status area.]{A screenshot of RePlay's status area. It reads, ``you clicked Repeat Grid'', and has the tool name ``Repeat Grid'' included in the search field.}
  \vspace{-0.2in}
\end{figure}


RePlay also displays contextual cues \cite{Ekstrand2011} on search results (\autoref{fig:replay-context_cues}). For each video, RePlay lists the three most-recently used applications that are mentioned in the video (\autoref{fig:replay-context_cues}a). This list is especially useful when users move between applications and want videos that mention both the current and recent applications. RePlay renders grey timeline markers to indicate moments where recently used tools are mentioned (\autoref{fig:replay-context_cues}b). Caption pop-ups italicize tool names.


\begin{figure}[b!]
\centering
\vspace{-0.2in}
  \includegraphics[width=\textwidth]{replay/figures/context_cues.png}
  \caption{RePlay displays contextual cues based on recent app and tool use. a) RePlay lists the three most recent apps that the video mentions. b) Grey markers indicate mentions of recently-used tools. In this example, the user recently used the ``text'' tool in Canva. Mousing over a marker shows a caption excerpt; clicking a marker plays the video. }~\label{fig:replay-context_cues}
%   \Description[A screenshot showing grey markers overlaid on a video result in RePlay.]{A screenshot showing grey markers overlaid on a video result in RePlay. The user has searched for ``how to export slides'' in Canva, and the top video result is titled ``How to create a canva presentation with PowerPoint''. The video timeline has grey dots on it, and the mouse is hovering over one. A pop-up shows a caption excerpt that says: ``...great I'm going to leave the text as it...'' with the word ``text'' italicized.}
  \vspace{-0.2in}
\end{figure}

RePlay's panel shows all results at the same time, allowing users to quickly skim multiple videos, and browse other results while one video plays. This ability to ``see inside'' multiple resources from a single page increases foraging efficiency \cite{Vermette2017, Glassman2016, Pavel2013}. 

Clicking the full-screen button in a video's bottom-right corner opens the video in a separate window that stays above all other windows while it is open, so that users can watch a video at a larger size when desired.

When the user switches to a new application and clicks a tool, RePlay automatically searches for the application name. This seeds the panel with app-relevant videos that give users a starting point.

\subsection{Detecting application context}
RePlay uses contextual information to augment a user's query and search within video results. The motivation for context-augmented search is to increase relevance, especially when users don't know what to ask for. However, if not done well, adding terms has the opposite effect: excluding relevant results and/or presenting irrelevant ones \cite{Finkelstein2002}. We tried several heuristics with RePlay; the current implementation includes the three most recent applications and tools.

\subsubsection{How to detect context?}
RePlay leverages \textsc{os} accessibility \textsc{api}s to detect every click. On each click event, RePlay retrieves the name of the click's source application, the type of element clicked, and the element's accessibility description (when present). If the element is a button, checkbox, text field, slider, or menu item, RePlay stores it as a recent tool and updates the status area and search field with the tool's name. If the user switched applications since their last click, RePlay updates its status area to reflect the new application, and resets its list of recent tools.

\subsubsection{Challenges with detecting context via accessibility}
Extracting accessibility text obviates the need for hard-coded knowledge about specific applications. The challenge of this system-wide approach is that despite platform accessibility guidelines, applications vary widely in what accessibility they offer and how \cite{Hurst2010}.

In Mac\-OS, RePlay can always extract the application name and menu items. Buttons and other interface elements have accessibility labels in many applications (\textit{e.g.,} Adobe \textsc{xd}, Microsoft Office, iMovie, Tableau, Sketch), but not in others (often long-existing software, \textit{e.g.,} Photoshop). Applications differ in which accessibility fields they support and what information is in what field. For example, tool names may be in the \textit{Title}, \textit{Help}, \textit{Description}, or \textit{Value} attribute. RePlay checks all four, preferring them in that order. RePlay also gathers accessibility information for websites, as long as browser accessibility access is enabled. It is by default in Safari, and as an option in Chrome. Many sites implement accessibility labels (\textit{e.g., } Canva, Wordpress, Sketchpad, Snappa, G Suite). Those that do not still include some information by default (\textit{e.g.}, text area contents via the \textit{Value} attribute). RePlay infers website names from the \textsc{url} and website title.

\subsection{Video search \& ranking}
RePlay leverages existing online video search engines to retrieve video results. It then finds and ranks relevant clips within these videos (\autoref{fig:replay-system}). RePlay's architecture requires no prior understanding of the applications or videos; it relies on captions for clip extraction and ranking. Video authors often talk about things they are doing and provide tips about tools; captions thus provide narrative information beyond the visual content in the video that can be useful for learners.

\begin{figure}[b!]
\centering
\vspace{-0.2in}
  \includegraphics[width=\textwidth]{replay/figures/replay-system.png}
  \caption{RePlay uses accessibility \textsc{api}s to detect user context, which it uses to augment the user's query and to search and rank videos. RePlay finds matching clips by searching video captions for the user's query and recent tool names. }~\label{fig:replay-system}
%   \Description[A diagram showing the different components of RePlay and how they communicate.]{A diagram showing the different components of RePlay and how they communicate. The MacOS Accessibility API detects user behavior from desktop software and sends this to RePlay. RePlay searches videos using the YouTube Data API. RePlay finds clips in these videos and displays them in the user interface.}
  \vspace{-0.2in}
\end{figure}

\subsubsection{Available data}
RePlay's current video corpus comprises all English videos on YouTube that have a caption track. Most do: YouTube auto-generates captions by default. We used YouTube for its popularity and captions; any video search engine with an \textsc{api} could be used. For example, Vimeo (\url{vimeo.com}) and Dailymotion (\url{dailymotion.com}) also provide \textsc{api} access to videos and captions.

\subsubsection{Searching}
Video search requires more steps than document search, because captions are obtained separately. This two-step search means that issuing multiple queries with context terms added (such as tool usage) like prior work \cite{Ekstrand2011} would be too slow. To speed responses, RePlay constructs and issues a single query concatenating the current application's name with the user's query. Leaving context terms out of the query also ensures that the user-provided search terms are not ``washed out''. RePlay queries YouTube and selects its top five video results that have English captions and mention the current application in any of the title, description, or captions (to avoid results that may contain other keywords but do not pertain to the current application).

\subsubsection{Finding clips and re-ranking videos}
Several techniques automatically extract instructional video clips from screencasts of software use. The dominant approach leverages application usage \cite{Grossman2010, Lafreniere2014, Chi2012, Wang2018}, requiring that the video be recorded in an instrumented version of the software. Alternatively, computer vision can detect tool-selection events \cite{Pongnumkul2011, Matejka2011}, even without prior knowledge about the specific software \cite{Banovic2012}. To be application-independent and embed online videos directly without waiting to download and process them, RePlay instead uses metadata and caption text to rank and segment videos. 

For each video result, RePlay divides its captions into 30-second segments, searching each for the queried keywords (with stop words removed) and names of the three most-recently used tools in the current application. It ranks all segments by the total number of keyword matches. To break ties it uses number of tool name matches. The highest-ranked segment determines the video's start time. Timeline markers denote the top ten segments: green for those with a query term; grey if only a tool is mentioned. RePlay re-orders the video results based on the total number of matching clips. To break ties it uses the total number of matching keywords within the clips.

Although automatic captions are far from perfect, we found them to be sufficient for searching in RePlay. Captions are already an approximation of what the demonstrator is doing, so despite some errors, they work well enough for identifying potentially relevant moments. Having any aids for navigating within videos is still an improvement over standard viewers. Still, future systems could allow viewers to easily correct errors as they watch to improve caption accuracy for future viewers.

\subsection{Implementation}
RePlay is implemented as a Mac\-OS application in Swift. It uses the Mac\-OS Accessibility \textsc{api} to extract information from input events. For both studies, RePlay used a whitelist to only detect clicks in certain applications and websites. A blacklist could be used instead; we explore this in the discussion.

When a search is triggered, RePlay queries the YouTube Data \textsc{api}'s \texttt{search} method, which returns an ordered list of video \textsc{id}s. For each \textsc{id}, RePlay checks if English captions are available using YouTube's \texttt{get\_video\_info} method. If they are, this method's response includes a \textsc{url} that RePlay follows to obtain subtitles in \textsc{xml}, with time stamps for every 5 to 10 words. A search that returns all new videos can take up to a few minutes to finish depending on network speed, due to the multiple requests needed to retrieve captions for each video. To speed up future searches, RePlay caches all retrieved captions locally (since they are pure text, this takes up little space). This could be further improved by proactively caching results for common search queries and buffering results as they come in. RePlay's video player is implemented in Javascript on a custom server; it embeds a YouTube video in an \texttt{iframe}, cues it to the given start time, and overlays timeline markers and pop-up captions. RePlay displays each video by loading this web player in a Swift \texttt{WKWebView} object. 
\input{critiquekit/4_deployments}
\input{critiquekit/5_experiments}
\section{General Discussion}
This chapter empirically investigated two techniques for scaffolding feedback: reusable feedback suggestions and adaptive guidance. This work complements this dissertation more broadly, highlighting the benefits of adaptive guidance for interfaces that support creative tasks. Here we discuss and synthesize the findings.  

\subsection{Generating Reusable Feedback Suggestions}
This work investigated whether suggestions and guidance can scaffold the feedback process. For this strategy to work, an eye towards reuse and adaptive feedback must be adopted. As Schön argues, experts may be most capable of recognizing common patterns and giving useful feedback \cite{Schon1983}. However, while feedback should be specific, underlying concepts can generalize across contexts. In the studies that used expert-generated feedback suggestions (DEP 1 and EXP 2), participants cited the same reason for why the suggestions were useful: as inspiration. Participants reported that the suggestions helped them find words for their thoughts or helped direct their attention to issues they did not originally notice. This suggests that reusable suggestions should focus attention to common issues rather than specific instances. Our approach demonstrates how expertise on creative work can be scaled by providing feedback on a few to apply to many \cite{Kulkarni2013}. This extends work on reusable feedback in coding and writing \cite{Brooks2014, Hartmann2010, Head2017} while keeping the human in the loop, enabling novices to learn and reuse expert insights. 

It is possible that more general suggestions can lead to less personalized feedback, particularly in abstract domains like visual design. We observed this in 7 of the 79 comments from the CritiqueKit condition in EXP 2, in which the four participants simply selected suggestions without further elaboration. A consideration for creating and presenting reusable suggestions is how these suggestions can be both general yet personal to be more helpful to the recipient. 

\subsection{What is the Best Way to Guide Feedback?}
Prior empirical work on feedback (\textit{e.g.}, Kulkarni \textit{et al.} \cite{Kulkarni2013} and Krause \textit{et al.} \cite{Krause2017}) has not compared static and adaptive suggestions. In this chapter, we found that people rarely used static suggestions and did not find them helpful; adaptive suggestions were used more and found more helpful. This reinforces prior work demonstrating that adaptive presentation of examples can improve learning \cite{Lee2010, Najar2014}. By presenting feedback suggestions that directly addressed missing characteristics of a reviewer's feedback, reviewers were prompted on where they could specifically improve, and explicitly shown examples of how to do so. 

The second experiment adapted feedback suggestions based on whether their feedback was categorized as specific, actionable, and/or justified. Though some of the prototype's categorizations were misleading or inaccurate (for example, the comment ``user flow is simple'' was categorized as ``Is Actionable'' because of the word ``use'', even though it lacks a concrete suggestion), participants still referenced the three categories when composing their comments. The guidance panel was useful as a reminder to include the attributes of good feedback in their comments. A more sophisticated method for categorization would likely be helpful, though our naïve approach performed reasonably well overall.

The guidance panel focused on three important attributes of good feedback. A consideration is to also provide guidance for emotional content in feedback, as emotional regulation is important to how learners perceive feedback \cite{Krause2017, Varlander2008}. In addition, other characteristics may also contribute to perceived helpfulness, such as complexity or novelty \cite{Krause2017}, that could be further explored through adaptive guidance.

\subsection{Creating Adaptive Feedback Interfaces}
In order for adaptive guidance to be most effective, the interface should be suitable for adaptation. In the two deployments and first experiment, the suggestions were not curated in any way: more than 1,400 comments were supplied as suggestions, but only 76 of these were reused by reviewers. Having more suggestions available was not beneficial because the suggestions were not sufficiently adaptable and were potentially irrelevant and difficult to browse. EXP 2 introduced a curated approach: experts provided the suggestions with generalizability in mind. Of the 47 suggestions created, 29 were reused. Though fewer suggestions were available, they were more general and adaptable, potentially making them more useful. 

Suggestion presentation shares many properties with search interfaces. Like with search, a good result needs to not only be in the set, but toward the top of the set \cite{Hearst2009Book}. The second experiment contained fewer suggestions, enabling easier search and browsing. Effective curation and display of suggestions should take into consideration the quality of feedback suggestions and how likely they are to be selected, potentially using frequency or some measure of generalizability as a signal. 

\section{Conclusion}
Looking across the deployments and experiments, adaptive suggestions and interactive guidance significantly improved feedback while static suggestions did not offer significant improvements. These techniques were embodied in the CritiqueKit system, used by 95 feedback providers and 336 recipients. Future work should examine applying other attributes of helpful feedback and further investigate how best to create, curate, and display adaptive suggestions. 

Much knowledge work features both underlying principles and context-specific knowledge of when and how to apply these principles. Potentially applicable feedback and review areas include domains as disparate as hiring and employee reviews, code reviews, product reviews, and reviews of academic papers, screenplays, business plans, and any other domain that blends context-specific creative choices with common genre structures. We hope that creativity support tools of all stripes will find value in the ideas and results presented here.


\section{Acknowledgements}
We thank Kandarp Khandwala and Janet Johnson for help rating feedback. This research was supported in part by a fellowship from Adobe Research.